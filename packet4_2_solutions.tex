\documentclass[letterpaper, twoside, 12pt]{book}
\usepackage{packet}


\begin{document}

\setcounter{chapter}{3}

\chapter{Packet 4.2: Sections 16.5-16.9}

\setcounter{chapter}{16}
\setcounter{section}{4}

\section{Curl and Divergence} %16.5

\begin{definition}
  The \textbf{curl} of a vector field $\vect F=\<P,Q,R\>$
  is given by the expression
  \[
    \text{curl }\vect{F}
      =
    \nabla \times \vect{F}
      =
    \left\<
      \frac{\partial}{\partial x},
      \frac{\partial}{\partial y},
      \frac{\partial}{\partial z}
    \right\>
      \times
    \<P,Q,R\>
      =
    \<R_y-Q_z,P_z-R_x,Q_x-P_y\>
  \]
\end{definition}

          \begin{problem}
            Prove that if $\vect{F}$ is conservative, then
            $\text{curl }\vect{F}=\vect{0}$.
          \end{problem}

          \begin{solution}
            Since $\vect{F}=\<P,Q,R\>$ is conservative,
            we know from a theorem in section 16.3 from the previous packet
            that $Q_x=P_y$, $P_z=R_x$, and $R_y=Q_z$. Therefore:
            \[
              \text{curl }\vect{F}
                =
              \<R_y-Q_z,P_z-R_x,Q_x-P_y\>
                =
              \<0,0,0\>=\vect{0}
            \]
          \end{solution}

\begin{remark}
  For a vector field $\vect{F}$ and direction $\vect{u}$,
  $(\text{curl }\vect{F})\cdot\vect{u}$ may be thought of as
  the tendency of $\vect{F}$ to ``spin'' counter-clockwise
  around $\vect{u}$.
\end{remark}

          \begin{problem}
            Compute the curl of $\<x+y,z^2-3,yz\>$ around the point
            $(2,0,-1)$.
          \end{problem}

          \begin{solution}
            Let $\vect{F}=\<P,Q,R\>=\<x+y,z^2-3,yz\>$. Then
            \[
              \text{curl }\vect{F}
                =
              \<R_y-Q_z,P_z-R_x,Q_x-P_y\>
            \]
            \[
                =
              \<(z)-(2z),(0)-(0),(0)-(1)\>
                =
              \<-z,0,-1\>
            \]

            By plugging in the point $(2,0,-1)$ we get
            \[
              \<-(-1),0,-1\>=\<1,0,-1\>
            \]
          \end{solution}

\begin{theorem}
  Green's Theorem may be rewritten in terms of curl as follows:
  \[
    \int_C \vect{F}\cdot\dvar{\vect{r}}
      =
    \iint_R (\text{curl }\vect{F}) \cdot \veck \dvar{A}
  \]
\end{theorem}

          \begin{problem}
            Prove the previous theorem.
          \end{problem}

          \begin{solution}
            Green's Theorem was given in section 4.1 to be
            \[
              \int_C \vect{F}\cdot\dvar{\vect{r}}
                =
              \iint_R Q_x-P_y \dvar{A}
            \]

            Since  \[
              \text{curl }\vect{F}
                =
              \<R_y-Q_z,P_z-R_x,Q_x-P_y\>
            \]
            we have that  \[
              \text{curl }\vect{F} \cdot \veck
                =
              \<R_y-Q_z,P_z-R_x,Q_x-P_y\> \cdot \<0,0,1\>
                =
              Q_x-P_y
            \]

            Therefore
            \[
              \int_C \vect{F}\cdot\dvar{\vect{r}}
                =
              \iint_R Q_x-P_y \dvar{A}
                =
              \iint_R \text{curl }\vect{F} \cdot \veck \dvar{A}
            \]
          \end{solution}

\begin{definition}
  The \textbf{divergence} of a vector field $\vect F=\<P,Q,R\>$
  is given by the expression
  \[
    \text{div }\vect{F}
      =
    \nabla \cdot \vect{F}
      =
    \left\<
      \frac{\partial}{\partial x},
      \frac{\partial}{\partial y},
      \frac{\partial}{\partial z}
    \right\>
      \cdot
    \<P,Q,R\>
      =
    P_x+Q_y+R_z
  \]
\end{definition}

          \begin{problem}
            Prove that the divergence of a curl vector field
            is always $0$. Put another way, show that
            $\text{div }(\text{curl }\vect{F})=0$.
          \end{problem}

          \begin{solution}
            We want to compute
            \[
              \text{div }(\text{curl }\vect{F})
                =
              \text{div }(\<R_y-Q_z,P_z-R_x,Q_x-P_y\>)
            \]

            Since divergence is the sum of the partial derivatives:
            \[
              \text{div }(\<R_y-Q_z,P_z-R_x,Q_x-P_y\>)
                =
              (R_{yx}-Q_{zx}) + (P_{zy}-R_{xy}) + (Q_{xz}-P_{yz})
            \]

            By regrouping, we find that
            \[
              (R_{yx}-Q_{zx}) + (P_{zy}-R_{xy}) + (Q_{xz}-P_{yz})
                =
              (R_{yx}-R_{xy}) + (P_{zy}-P_{yz}) + (Q_{xz}-Q_{zx})
                =
              0
            \]
          \end{solution}

\begin{remark}
  Divergence measures the tendency of a vector field to diverge away
  from a point.
\end{remark}

          \begin{problem}
            Compute the divergence of $\<x+y,z^2-3,yz\>$ away from the point
            $(2,0,-1)$.
          \end{problem}

          \begin{solution}
            Let $\vect{F}=\<P,Q,R\>=\<x+y,z^2-3,yz\>$.
            We compute divergence as follows:
            \[
              \text{div }\vect{F}
                =
              P_x+Q_y+R_z
                =
              (1)+(0)+(y)
                =
              1+y
            \]

            Plugging in the point $(2,0,-1)$ gives
            \[
              1+(0) = 1
            \]
          \end{solution}

\begin{definition}
  The \textbf{flux} of a velocity vector field $\vect{F}$ across a closed
  curve $C$ is given by
  \[
    \int_C \vect{F}\cdot\vect{n}\dvar{s}
  \]
  where $\vect n$ yields outward unit normal vectors to $C$.
\end{definition}

\begin{remark}
  Flux measures the tendency of a vector field to flow outward from
  a closed and bounded region (or inward if the flux is negative).
\end{remark}

\begin{theorem}
  Green's Theorem may be rewritten in terms of divergence as follows:
  \[
    \int_C \vect{F}\cdot\vect{n}\dvar{s}
      =
    \iint_R \text{div }\vect{F} \dvar{A}
  \]
\end{theorem}

          \begin{problem}
            Compute the flux of the velocity vector field
            $\<x+y,x^2+y^2\>$ across the boundary of the unit square.
          \end{problem}

          \begin{solution}
            Let $\vect{F}=\<x+y,x^2+y^2\>$, so then its divergence is
            $\text{div }\vect{F}=(1)+(2y)=1+2y$.
            By the previous theorem, the flux may be computed by
            \[
              \int_C \vect{F}\cdot\vect{n}\dvar{s}
                =
              \iint_R \text{div }\vect{F} \dvar{A}
                =
              \iint_R 1+2y \dvar{A}
            \]

            Since $R$ is the unit square,
            \[
              \iint_R 1+2y \dvar{A}
                =
              \int_0^1\int_0^1 1+2y \dvar{y}\dvar{x}
                =
              \int_0^1 2\dvar{x}
                =
              2
            \]
          \end{solution}

\section{Parametric Surfaces} %16.6

\begin{remark}
  Just like a curve may be parameterized by $\vect{r}(t)$
  for an interval $a\leq t\leq b$, a surface may be parameterized by
  $\vect{r}(u,v)$ for a region $R$ in the $uv$ plane.
\end{remark}

\begin{theorem}
  Following are some common surface parameterizations.
  \begin{itemize}
    \item The surface $z=f(x,y)$ may be parametrized by
      \[
        \vect{r}(x,y) = \<x,y,f(x,y)\>
      \]
    \item A surface determined by a cylindrical coordinate equation may
    be parametrized by substituting into
      \[
        \vect{r} = \<r\cos\theta, r\sin\theta, z\>
      \]
    \item A surface determined by a spherical coordinate equation may
    be parametrized by substituting into
      \[
        \vect{r} =
        \<\rho\sin\phi\cos\theta,
        \rho\sin\phi\sin\theta,
        \rho\cos\phi \>
      \]
  \end{itemize}
\end{theorem}

          \begin{problem}
            Find a parameterization from the $xy$ plane to the
            plane $2x-y+z=7$ in $xyz$ space.
          \end{problem}

          \begin{solution}
            The surface $z=f(x,y)$ may be parametrized by
              \[
                \vect{r}(x,y) = \<x,y,f(x,y)\>
              \]
            So we can rewrite the surface as $z=7-2x+y=f(x,y)$, and use the
            parameterization
              \[
                \vect{r}(x,y) = \<x,y,7-2x+y\>
              \]
          \end{solution}

          \begin{problem}
            Find the parameterization from the rectangle $0\leq z\leq 3$
            and $0\leq\theta\leq2\pi$ to the conical surface $z=\sqrt{x^2+y^2}$
            below the plane $z=3$ in $xyz$ space. (Hint: find the cylindrical
            coordinate equation for the surface.)
          \end{problem}

          \begin{solution}
            The surface $z=\sqrt{x^2+y^2}$ has cylindrical coordinate
            equation $z=\sqrt{r^2}=r$.
            Substituting that into the cylindrical coordinate transformation
            \[
              \vect{r}(r,\theta,z)
                =
              \<r\cos\theta,r\sin\theta,z\>
            \]
            we get
            \[
              \vect{r}(\theta,z)
                =
              \<z\cos\theta,z\sin\theta,z\>
            \]

            Since the cone revolves fully around the $z$-axis,
            $0\leq\theta\leq2\pi$, and we already know $0\leq z\leq 3$.
          \end{solution}

          \begin{problem}
            Find the parameterization from the rectangle $0\leq\phi\leq\pi$
            and $0\leq\theta\leq2\pi$ to the spherical surface
            $x^2+y^2+z^2=9$ in $xyz$ space. (Hint: find the spherical
            coordinate equation for the surface.)
          \end{problem}

          \begin{solution}
            The surface $x^2+y^2+z^2=9$ has spherical coordinate
            equation $\rho=3$.
            Substituting that into the spherical coordinate transformation
            \[
              \vect{r}(\rho,\phi,\theta)
                =
              \<\rho\sin\phi\cos\theta, \rho\sin\phi\sin\theta, \rho\cos\phi\>
            \]
            we get
            \[
              \vect{r}(\phi,\theta)
                =
              \<3\sin\phi\cos\theta, 3\sin\phi\sin\theta, 3\cos\phi\>
            \]

            Since the sphere revolves fully around the $z$-axis,
            $0\leq\theta\leq2\pi$, and since it includes points on the positive
            and negative $z$-axis, $0\leq\phi\leq\pi$.
          \end{solution}


\section{Surface Integrals} %16.7

\begin{definition}
  The \textbf{surface integral} of a function $f(x,y,z)$ over a surface
  $S$ in $xyz$ space is given by
  \[
    \iint_S f(\vect{r})\dvar{\sigma}
      =
    \iint_R f(\vect{r}(u,v))|\vect{r}_u\times\vect{r}_v|\dvar{A}
  \]
  where $\vect{r}(u,v)$ is a parameterization from the region $R$ in
  the $uv$ plane to the surface $S$.
\end{definition}

\begin{theorem}
  The surface area of $S$ is given by
  \[
    \iint_S \dvar{\sigma} = \iint_S 1\dvar{\sigma}
  \]
\end{theorem}

          \begin{problem}
            Use the parameterization
            \[
              \vect{r}(\phi,\theta)
                =
              \<
                \sin\phi\cos\theta,
                \sin\phi\sin\theta,
                \cos\phi
              \>
            \]
            from $0\leq\phi\leq\pi,0\leq\theta\leq2\pi$ to the unit
            sphere to show that the surface area of the unit sphere
            is $4\pi$. (Note that this matches the formula $SA=4\pi r^2$ used
            in high school geometry.)
          \end{problem}

          \begin{solution}
            The surface area of $S$ is given by
            \[
              \iint_S \dvar{\sigma}
                =
              \iint_R |\vect{r}_\phi\times\vect{r}_\theta|\dvar{A}
            \]

            We may compute the cross product:
            \[
              \vect{r}_\phi\times\vect{r}_\theta
                =
              \<\cos\phi\cos\theta,\cos\phi\sin\theta,-\sin\phi\>
                \times
              \<-\sin\phi\sin\theta,\sin\phi\cos\theta,0\>
            \]
            \[
                =
              \<\sin^2\phi\cos\theta,-\sin^2\phi\sin\theta,\sin\phi\cos\phi\>
            \]

            And its magnitude is then
            \[
              |\vect{r}_\phi\times\vect{r}_\theta|
                =
              |\sin\phi|
            \]
            but we can drop the absolute values since $0\leq\phi\leq\pi$.

            Using the bounds and plugging in, we get
            \[
              \iint_R |\vect{r}_\phi\times\vect{r}_\theta|\dvar{A}
                =
              \int_0^{2\pi}\int_0^\pi \sin\phi \dvar{\phi}\dvar{\theta}
                =
              \int_0^{2\pi} 2\dvar{\theta}
                =
              4\pi
            \]
          \end{solution}

          \begin{problem}
            Show that the area of the parallelogram with vertices $(0,0,0)$,
            $(2,1,2)$, $(0,2,-1)$, and $(2,3,1)$ is $3\sqrt{5}$ using a surface
            integral.
            (Hint: use $\vect{r}(u,v)=\<2u,u+2v,2u-v\>$.)
          \end{problem}

          \begin{solution}
            Note that for the parameterization $\vect{r}(u,v)=\<2u,u+2v,2u-v\>$,
            $\vect{r}(0,0)=\<0,0,0\>$, $\vect{r}(1,0)=\<2,1,2\>$,
            $\vect{r}(0,1)=\<0,2,-1\>$, and $\vect{r}(1,1)=\<2,3,1\>$. So
            it maps the unit square onto the given parallelogram.

            So we want to compute
            \[
              \iint_S \dvar{\sigma}
                =
              \iint_R |\vect{r}_u\times\vect{r}_v|\dvar{A}
                =
              \int_0^1\int_0^1 |\vect{r}_u\times\vect{r}_v|\dvar{v}\dvar{u}
            \]

            We compute the cross product as follows:
            \[
              \vect{r}_u\times\vect{r}_v
                =
              \<2,1,2\>\times\<0,2,-1\>
                =
              \<-5,2,4\>
            \]
            \[
              |\vect{r}_u\times\vect{r}_v|
                =
              \sqrt{25+4+16}
                =
              \sqrt{45}
                =
              3\sqrt{5}
            \]

            It follows that
            \[
              \int_0^1\int_0^1 3\sqrt{5}\dvar{v}\dvar{u}
                =
              \int_0^1 3\sqrt{5}\dvar{u}
                =
              3\sqrt{5}
            \]
          \end{solution}

\begin{definition}
  An \textbf{orientation} of a surface is a continuous unit
  vector field normal to the surface.
\end{definition}

\begin{remark}
  Orienting a surface is akin to choosing one side or another of the surface.
\end{remark}

\begin{remark}
  Examples of non-orientable surfaces are the Mobi\"us strip and Klein bottle.
\end{remark}

\begin{definition}
  The \textbf{surface integral} of a vector field $\vect{F}$ over an
  oriented surface $S$ in $xyz$ space is given by
  \[
    \iint_S \vect{F}\cdot\dvar{\vect\sigma}
      =
    \iint_S \vect{F}\cdot\vect{n}\dvar\sigma
      =
    \iint_R \vect{F}\cdot(\vect{r}_u\times\vect{r}_v)\dvar{A}
  \]
  where $\vect n$ is the orientation of the surface and giving
  its orientation, and
  $\vect{r}(u,v)$ is an appropriate parameterization from the region $R$ in
  the $uv$ plane to the surface $S$.
\end{definition}

\begin{definition}
  The \textbf{flux} across a closed oriented surface (such as
  the boundary of a solid) is given by
  \[
    \iint_S \vect{F}\cdot\dvar{\vect\sigma}
  \]
\end{definition}

          \begin{problem}
            Use the parameterization
            \[
              \vect{r}(\phi,\theta)
                =
              \<
                3\sin\phi\cos\theta,
                3\sin\phi\sin\theta,
                3\cos\phi
              \>
            \]
            from $0\leq\phi\leq\pi,0\leq\theta\leq2\pi$ to the sphere
            $x^2+y^2+z^2=9$ to prove that the flux across it for the
            vector field $\<x,y,z\>$ is
            \[
              \int_0^{2\pi}\int_0^\pi 27\sin\phi \dvar{\phi}\dvar{\theta}
            \]
          \end{problem}

          \begin{solution}
            We want to compute
            \[
              \iint_S \vect{F}\cdot\dvar{\vect\sigma}
                =
              \iint_R \vect{F}\cdot(\vect{r}_\phi\times\vect{r}_\theta)\dvar{A}
                =
              \int_0^{2\pi}\int_0^\pi
              \vect{F}\cdot(\vect{r}_\phi\times\vect{r}_\theta)
              \dvar{\phi}\dvar{\theta}
            \]

            Using the given parametrization,
            \[
              \vect{F}=\<x,y,z\>=
              \<
                3\sin\phi\cos\theta,
                3\sin\phi\sin\theta,
                3\cos\phi
              \>
            \]
            and the cross product is
            \[
              \vect{r}_\phi\times\vect{r}_\theta
                =
              \<
                3\cos\phi\cos\theta,
                3\cos\phi\sin\theta,
                -3\sin\phi
              \>
                \times
              \<
                -3\sin\phi\sin\theta,
                3\sin\phi\cos\theta,
                0
              \>
            \]
            \[
                =
              \<
                9\sin^2\phi\cos\theta,
                9\sin^2\phi\sin\theta,
                9\sin\phi\cos\phi
              \>
            \]

            So their dot product gives:
            \[
              \int_0^{2\pi}\int_0^\pi
              \vect{F}\cdot(\vect{r}_\phi\times\vect{r}_\theta)
              \dvar{\phi}\dvar{\theta}
                =
              \int_0^{2\pi}\int_0^\pi
              27\sin^3\phi\cos^2\theta +
              27\sin^3\phi\sin^2\theta +
              27\sin\phi\cos^2\phi
              \dvar{\phi}\dvar{\theta}
            \]
            \[
                =
              \int_0^{2\pi}\int_0^\pi
              27\sin^3\phi +
              27\sin\phi\cos^2\phi
              \dvar{\phi}\dvar{\theta}
                =
              \int_0^{2\pi}\int_0^\pi
              27\sin\phi
              \dvar{\phi}\dvar{\theta}
            \]
          \end{solution}


\section{Stokes' Theorem}%16.8

\begin{theorem}
  Let $S$ be a surface with orientation $\vect{n}$
  and with boundary $C$ oriented counter-clockwise with respect to $\vect{n}$.
  Then
  \[
    \iint_S \text{curl }\vect{F}\cdot\dvar{\vect\sigma}
      =
    \int_C \vect{F}\cdot\dvar{\vect{r}}
  \]
\end{theorem}

          \begin{problem}
            Let $S$ be the upper hemisphere $z=\sqrt{1-x^2-y^2}$. Use
            Stokes' Theorem to prove that
            \[
              \iint_S \<2y,2z,2x\>\cdot\dvar{\vect\sigma}
                =
              \int_0^{2\pi} \cos^3(t) \dvar{t}
            \]
            (Hint: what's the curl of $\<z^2,x^2,y^2\>$?).
          \end{problem}

          \begin{solution}
            Following the hint, the curl of $\<z^2,x^2,y^2\>$ may be computed
            to be:
            \[
              \text{curl }\<z^2,x^2,y^2\>
                =
              \<2y-0,2z-0,2x-0\>
            \]

            So it follows from Stokes' Theorem that
            \[
              \iint_S \<2y,2z,2x\>\cdot\dvar{\vect\sigma}
                =
              \iint_S \text{curl }\<z^2,x^2,y^2\>\cdot\dvar{\vect\sigma}
                =
              \int_C \<z^2,x^2,y^2\>\cdot\dvar{\vect r}
            \]

            $C$ is the boundary of $S$, the circle $x^2+y^2=1$ in the
            plane $z=0$. It has a parametrization
            \[
              \vect{r}(t) = \<\cos t,\sin t, 0\>
            \]
            for $0\leq t\leq 2\pi$ and derivative
            \[
              \frac{d\vect r}{dt} = \<-\sin t,\cos t, 0\>
            \]

            So the line integral may be rewritten as
            \[
              \int_C \<z^2,x^2,y^2\>\cdot\dvar{\vect r}
                =
              \int_C \<z^2,x^2,y^2\>\cdot\frac{d\vect r}{dt}\dvar{t}
                =
              \int_0^{2\pi}
              \<0^2,(\cos t)^2,(\sin t)^2\>
              \cdot
              \<-\sin t,\cos t, 0\>
              \dvar{t}
            \]
            \[
                =
              \int_0^{2\pi} 0 + \cos^3(t) + 0 \dvar{t}
                =
              \int_0^{2\pi} \cos^3(t) \dvar{t}
            \]
          \end{solution}


\section{Divergence Theorem}%16.8

\begin{theorem}
  Let $S$ be the boundary of a solid $D$ oriented outwards.
  Then
  \[
    \iint_S \vect{F}\cdot\dvar{\vect\sigma}
      =
    \iiint_D \text{div }\vect{F}\dvar{V}
  \]
\end{theorem}

          \begin{problem}
            Let $S$ be the boundary of the unit cube in $xyz$ space.
            Use the Divergence Theorem to prove that
            \[
              \iint_S \<x+y,y^2+z^2,z^3+x^3\>\cdot\dvar{\vect\sigma}
                =
              \int_0^1\int_0^1\int_0^1 1+2y+3z^2 \dvar{z}\dvar{y}\dvar{x}
            \]
          \end{problem}

          \begin{solution}
            By the Divergence Theorem:
            \[
              \iint_S \<x+y,y^2+z^2,z^3+x^3\>\cdot\dvar{\vect\sigma}
                =
              \iiint_D \text{div }\<x+y,y^2+z^2,z^3+x^3\>\dvar{V}
            \]
            \[
                =
              \int_0^1\int_0^1\int_0^1 (1)+(2y)+(3z^2)\dvar{z}\dvar{y}\dvar{x}
            \]
          \end{solution}

\section{A small remark and puzzle}

\begin{remark}
  Using derivatives, gradients, curl, and divergence, we may observe that
  several kinds of integrals may be evaluated by observing how the
  integrand behaves on the boundary of the domain of integration, and
  vice versa.
  \[
    \int_{[a,b]} f'(x)\dvar{x} = [f(x)]_a^b
  \]
  \[
    \int_C \nabla f\cdot \dvar{\vect r} = [f(P)]_A^B
  \]
  \[
    \iint_R \text{div }\vect{F} \dvar{A}
      =
    \int_C \vect{F}\cdot\vect{n}\dvar{s}
  \]
  \[
    \iint_R Q_x-P_y \dvar{A}
      =
    \int_C \<P,Q\>\cdot\dvar{\vect r}
  \]
  \[
    \iint_S \text{curl }\vect{F}\cdot\dvar{\vect\sigma}
      =
    \int_C \vect{F}\cdot\dvar{\vect{r}}
  \]
  \[
    \iiint_D \text{div }\vect{F}\dvar{V}
      =
    \iint_S \vect{F}\cdot\dvar{\vect\sigma}
  \]
\end{remark}

\begin{problem}
  (OPTIONAL)
  This has nothing to do with the above remark, but here's a puzzle for
  reading this far.

  Wayne Brady is hosting a gameshow, and you've
  been called down from the audience to attempt to win fabulous prizes.
  Wayne gives you the choice of three doors: $A$, $B$, and $C$. He asks
  you to choose a door, explaining that only one of the three doors holds
  a prize behind it.

  After you choose, Wayne opens one of the doors that you didn't choose to
  reveal nothing behind it. He then offers you the opportunity to switch
  your door with the other unopened door, after which you will immediately
  be given whatever is behind it. Should you stick with your initial
  guess, or should you switch, or does it even matter? Why?
\end{problem}

          \begin{solution}
            Figure it out yourself. :-)
          \end{solution}


\end{document}