\documentclass[letterpaper, twoside, 12pt]{book}
\usepackage{packet}


\begin{document}

\setcounter{chapter}{1}

\chapter{Part 1: Sections 13.3-13.4}

\setcounter{chapter}{13}
\setcounter{section}{2}

\section{Arc Length and Curvature}

          \begin{problem}
            Let $\vect{r}(t)=\<6t, t^3, 3t^2\>$. Use the lengths of
            the line segments
            connecting $\vect{r}(0)$, $\vect{r}(1)$, $\vect{r}(2)$,
            and $\vect{r}(3)$ to approximate the length of the curve
            from $t=0$ to $t=3$.
          \end{problem}

          \begin{solution}

          \end{solution}

\begin{definition}
Let $\harpvec{r}(t) = \<f(t),g(t),h(t)\>$ be a vector function.
Then the \textbf{arclength} or \textbf{length} of the curve given by
$\harpvec{r}(t)$ from $t=a$ to $t=b$ is
\[
  L
    =
  \int_a^b
  \left|
    \lim_{\Delta{t}\to0}
    \frac{\vect{r}(t+\Delta{t})-\vect{r}(t)}{\Delta{t}}
  \right|
  \dvar{t}
  =
  \int_a^b |\vect{r}'(t)| \dvar{t}
\]
\end{definition}

          \begin{problem}
            Find the length of the curve given by
            $\vect{r}(t)=\<6t, t^3, 3t^2\>$
            from $t=0$ to $t=3$.
          \end{problem}

          \begin{solution}

          \end{solution}

\begin{definition}
Let $s(t)$ be the \textbf{arclength function/parameter} representing the
length of a curve from the point given by
$\harpvec{r}(0)$ to the point given by $\harpvec{r}(t)$.
(Assume $s(t)<0$ for $t<0$.)
\end{definition}

\begin{theorem}
The arclength function $s(t)$ is given by the definite integral
\[
  s(t)
    =
  \int_0^t |\vect{r}'(\tau)| \dvar{\tau}
\]
\end{theorem}

\begin{theorem}
The derivative of the arclength function gives the lengths of
the tangent vectors given by the derivative of the position function:
\[
  \frac{ds}{dt} = \left|\frac{d\vect{r}}{dt}\right|
\]
\end{theorem}

          \begin{problem}
            Compute $s(t)$ for $\vect{r}(t)=\<6t, t^3, 3t^2\>$,
            and use it to find the arclength parameter corresponding
            to $t=-2$.
          \end{problem}

          \begin{solution}

          \end{solution}

          \begin{problem}
            Find the length of an arc of the circular helix with
            vector equation
            $\vect{r}(t) = \<\cos(t),\sin(t),t\>$
            from $(1,0,0)$ to $(1,0,2\pi)$.
          \end{problem}

\begin{definition}
  The \textbf{unit tangent vector} $\vect{T}$ to a curve $\vect{r}$ is the
  direction of the derivative $\vect{r}'(t)=\frac{d\vect{r}}{dt}$.
\end{definition}

\begin{theorem}
  \[
    \vect{T} = \frac{d\vect{r}/dt}{|d\vect{r}/dt|} = \frac{d\vect{r}}{ds}
  \]
\end{theorem}

\begin{definition}
  The \textbf{curvature} $\kappa$ of a curve $C$ at a given point is
  the magnitude of the rate of change of $\vect{T}$ with respect to
  arclength $s$.
\end{definition}

\begin{theorem}
  \[
    \kappa
      =
    \left|
    \frac{d\vect{T}}{ds}
    \right|
      =
    \left|
    \frac{1}{ds/dt}
    \frac{d\vect{T}}{dt}
    \right|
      =
    \frac{1}{|d\vect{r}/dt|}
    \left|
      \frac{d\vect{T}}{dt}
    \right|
  \]
\end{theorem}

\begin{theorem}
  An alternate formula for curvature is given by
  \[
    \kappa =
    \frac{|\vect{r}'(t)\times\vect{r}''(t)|}{|\vect{r}'(t)|^3}
  \]
\end{theorem}

          \begin{problem}
            Prove that the helix given by the vector equation
            $\vect{r}(t) = \<\cos(t),\sin(t),t\>$
            has constant curvature.
          \end{problem}

          \begin{solution}

          \end{solution}

          \begin{problem}
            (OPTIONAL)
            Prove that the alternate formula for curvature is
            accurate by showing
            \[
              \frac{1}{|d\vect{r}/dt|}
              \left|
                \frac{d\vect{T}}{dt}
              \right|
                =
              \frac{|\vect{r}'(t)\times\vect{r}''(t)|}{|\vect{r}'(t)|^3}
            \]
            (Some of the solution has been provided.)
          \end{problem}

          \begin{solution}
            Begin by observing that
            $
              \vect{r}'
                =
              \left|\frac{d\vect{r}}{dt}\right|\vect{T}
                =
              \frac{ds}{dt}\vect{T}
            $, and by the product rule it follows that
            $
              \vect{r}''
                =
              \frac{d^2s}{dt^2}\vect{T} + \frac{ds}{dt}\vect{T}'
            $.

            (...)

            % (Continue this argument by taking the cross-product of
            % $\vect{r}'$ and $\vect{r}''$, simplifying by using the fact that
            % $\vect{v}\times\vect{v}=\vect{0}$, then taking its magnitude
            % and simplifying using the fact that $|\vect{T}|=1$ and
            % $\vect{T},\vect{T}'$ are perpendicular (why?).
            % You should end up with
            % $\left(\frac{ds}{dt}\right)^2\left|\vect{T}'\right|$, which can
            % be used with $\frac{ds}{dt}=|\vect{r}'|$ to finish the proof.)
          \end{solution}

\begin{definition}
  The \textbf{unit normal vector} $\vect{N}$ to a curve $\vect{r}$ is the
  direction of the derivative of the unit tangent vector
  $\vect{T}'(t)=\frac{d\vect{T}}{dt}$.
  (By definition, this vector points into the direction of the curve.)
\end{definition}

\begin{theorem}
  \[
    \vect{N} = \frac{\vect{T}'}{|\vect{T}'|}
  \]
\end{theorem}

          \begin{problem}
            Prove that $\vect{N}$ is actually normal to the curve by
            using a theorem from a previous section. (Hint: $|\vect{T}|=1$.)
          \end{problem}

          \begin{solution}

          \end{solution}

\begin{definition}
  The \textbf{binormal vector} $\harpvec{B}$ is the direction
  normal to both $\harpvec{T}$ and $\harpvec{N}$ according to
  the right-hand rule.
\end{definition}

\begin{theorem}
  \[
    \vect{B}=\vect{T}\times\vect{N}
  \]
\end{theorem}

          \begin{problem}
            Prove that $\vect{T}\times\vect{N}$ is a unit vector.
          \end{problem}

          \begin{solution}

          \end{solution}

% % \begin{figure}[H]
% % \centering
% % \includegraphics[width=1.5in]{TangentNormalBinormal.pdf}
% % \label{TangentNormalBinormal}
% % \caption{Frenet-Serret Equations}
% % \end{figure}


% \begin{definition}[Normal Plane\index{Normal Plane}]
% The \textbf{Normal Plane} is defined by $\harpvec{N}$ and $\harpvec{B}$.  It represents all vectors that are perpendicular to $\harpvec{T}.$
% \end{definition}

% \begin{definition}[Osculating (or Kissing) Plane\index{Osculating Plane}]
% The \textbf{osculating plane} is the plane defined by $\harpvec{T}$ and $\harpvec{N}$.  It represents the plane that most closely fits the curve at that point.
% \end{definition}

% \begin{definition}[Osculating Circle\index{Osculating Circle}]
% The \textbf{osculating circle} is the circle in the osculating plane that has the same tangent as the curve $C$ at the point $P$ and lies on the concave side of $C$ with radius $\rho = \frac{1}{\kappa}.$
% \end{definition}

% % \begin{figure}[H]
% % \centering
% % \includegraphics[width=1.5in]{OsculatinCircle.pdf}
% % \label{OsculatingCircle}
% % \caption{Osculating Circle}
% % \end{figure}

% \begin{problem}
% Find the equation of the normal plane and the osculating plane of the helix $\harpvec{r}(t) = \threevec{\cos(t)}{\sin(t)}{t}$ at the point $P = \left(0,1,\frac{\pi}{2}\right).$
% \end{problem}

% \vfill

% \noindent Suggested Exercises: Section $13.3: 1-6, 17-25, 49,50.$

% \newpage

% \section{Motion in Space, Velocity, and Acceleration}

% \begin{definition}[Velocity, Acceleration, \& Speed\index{Velocity}\index{Speed}\index{Acceleration}]
% Let $\harpvec{r}(t)$ be a position vector for a particle at time $t$.  Then the \textbf{velocity} of the particle at time $t$ is $\harpvec{v}(t) = \harpvec{r}^\prime(t)$ and the \textbf{acceleration} of the particle at time $t$ is $\harpvec{a}(t) = \harpvec{v}\prime(t) = \harpvec{r}^{\prime\prime}(t).$  The speed of the particle at time $t$ is defined by $\left|\harpvec{v}(t)\right| = \left|\harpvec{r}^\prime(t)\right|.$
% \end{definition}

% \begin{problem}
% Given a position vector $\harpvec{r}(t) = \twovec{t^3}{t^2},$ find the velocity, speed, and acceleration at $t = 1$.
% \end{problem}

% \vfill

% \begin{problem}
% A moving particle starts at an initial position $\harpvec{r}(0) = \threevec{1}{0}{0}$, has initial velocity $\harpvec{v}(0) = \threevec{1}{-1}{1}$, and acceleration is defined by $\harpvec{a}(t) = \threevec{4t}{6t}{1}.$  Find the particle's position and velocity at time $t.$
% \end{problem}

% \vfill

\end{document}