\documentclass[letterpaper, twoside, 12pt]{book}
\usepackage{packet}


\begin{document}

\setcounter{chapter}{3}

\chapter{Packet 4.2: Sections 16.5-16.9}

\setcounter{chapter}{16}
\setcounter{section}{4}

\section{Curl and Divergence} %16.5

\begin{definition}
  The \textbf{curl} of a vector field $\vect F=\<P,Q,R\>$
  is given by the expression
  \[
    \text{curl }\vect{F}
      =
    \nabla \times \vect{F}
      =
    \left\<
      \frac{\partial}{\partial x},
      \frac{\partial}{\partial y},
      \frac{\partial}{\partial z}
    \right\>
      \times
    \<P,Q,R\>
      =
    \<R_y-Q_z,P_z-R_x,Q_x-P_y\>
  \]
\end{definition}

          \begin{problem}
            Prove that if $\vect{F}$ is conservative, then
            $\text{curl }\vect{F}=\vect{0}$.
          \end{problem}

          \begin{solution}

          \end{solution}

          \begin{contributors}

          \end{contributors}

\begin{remark}
  For a vector field $\vect{F}$ and direction $\vect{u}$,
  $(\text{curl }\vect{F})\cdot\vect{u}$ may be thought of as
  the tendency of $\vect{F}$ to ``spin'' counter-clockwise
  around $\vect{u}$.
\end{remark}

          \begin{problem}
            Compute the curl of $\<x+y,z^2-3,yz\>$ around the point
            $(2,0,-1)$.
          \end{problem}

          \begin{solution}

          \end{solution}

          \begin{contributors}

          \end{contributors}

\begin{theorem}
  Green's Theorem may be rewritten in terms of curl as follows:
  \[
    \int_C \vect{F}\cdot\dvar{\vect{r}}
      =
    \iint_D (\text{curl }\vect{F}) \cdot \veck \dvar{A}
  \]
\end{theorem}

          \begin{problem}
            Prove the previous theorem.
          \end{problem}

          \begin{solution}

          \end{solution}

          \begin{contributors}

          \end{contributors}

\begin{definition}
  The \textbf{divergence} of a vector field $\vect F=\<P,Q,R\>$
  is given by the expression
  \[
    \text{div }\vect{F}
      =
    \nabla \cdot \vect{F}
      =
    \left\<
      \frac{\partial}{\partial x},
      \frac{\partial}{\partial y},
      \frac{\partial}{\partial z}
    \right\>
      \cdot
    \<P,Q,R\>
      =
    P_x+Q_y+R_z
  \]
\end{definition}

          \begin{problem}
            Prove that the divergence of a curl vector field
            is always $0$. Put another way, show that
            $\text{div }(\text{curl }\vect{F})=0$.
          \end{problem}

          \begin{solution}

          \end{solution}

          \begin{contributors}

          \end{contributors}

\begin{remark}
  Divergence measures the tendency of a vector field to diverge away
  from a point.
\end{remark}

          \begin{problem}
            Compute the divergence of $\<x+y,z^2-3,yz\>$ away from the point
            $(2,0,-1)$.
          \end{problem}

          \begin{solution}

          \end{solution}

          \begin{contributors}

          \end{contributors}

\begin{definition}
  The \textbf{flux} of a velocity vector field $\vect{F}$ across a closed
  curve $C$ is given by
  \[
    \int_C \vect{F}\cdot\vect{n}\dvar{s}
  \]
  where $\vect n$ yields outward unit normal vectors to $C$.
\end{definition}

\begin{remark}
  Flux measures the tendency of a vector field to flow outward from
  a closed and bounded region (or inward if the flux is negative).
\end{remark}

\begin{theorem}
  Green's Theorem may be rewritten in terms of divergence as follows:
  \[
    \int_C \vect{F}\cdot\vect{n}\dvar{s}
      =
    \iint_D \text{div }\vect{F} \dvar{A}
  \]
\end{theorem}

          \begin{problem}
            Compute the flux of the velocity vector field
            $\<x+y,x^2+y^2\>$ across the boundary of the unit square.
          \end{problem}

          \begin{solution}

          \end{solution}

          \begin{contributors}

          \end{contributors}

\section{Parametric Surfaces} %16.6

\begin{remark}
  Just like a curve may be parameterized by $\vect{r}(t)$
  for an interval $a\leq t\leq b$, a surface may be parameterized by
  $\vect{r}(u,v)$ for a region $R$ in the $uv$ plane.
\end{remark}

\begin{theorem}
  Following are some common surface parameterizations.
  \begin{itemize}
    \item The surface $z=f(x,y)$ may be parametrized by
      \[
        \vect{r}(x,y) = \<x,y,f(x,y)\>
      \]
    \item A surface determined by a cylindrical coordinate equation may
    be parametrized by substituting into
      \[
        \vect{r} = \<r\cos\theta, r\sin\theta, z\>
      \]
    \item A surface determined by a spherical coordinate equation may
    be parametrized by substituting into
      \[
        \vect{r} =
        \<\rho\sin\phi\cos\theta,
        \rho\sin\phi\sin\theta,
        \rho\cos\phi \>
      \]
  \end{itemize}
\end{theorem}

          \begin{problem}
            Find a parameterization from the $xy$ plane to the
            plane $2x-y+z=7$ in $xyz$ space.
          \end{problem}

          \begin{solution}

          \end{solution}

          \begin{contributors}

          \end{contributors}

          \begin{problem}
            Find the parameterization from the rectangle $0\leq z\leq 3$
            and $0\leq\theta\leq2\pi$ to the conical surface $z=\sqrt{x^2+y^2}$
            below the plane $z=3$ in $xyz$ space. (Hint: find the cylindrical
            coordinate equation for the surface.)
          \end{problem}

          \begin{solution}

          \end{solution}

          \begin{contributors}

          \end{contributors}

          \begin{problem}
            Find the parameterization from the rectangle $0\leq\phi\leq\pi$
            and $0\leq\theta\leq2\pi$ to the spherical surface
            $x^2+y^2+z^2=9$ in $xyz$ space. (Hint: find the spherical
            coordinate equation for the surface.)
          \end{problem}

          \begin{solution}

          \end{solution}

          \begin{contributors}

          \end{contributors}


\section{Surface Integrals} %16.7

\begin{definition}
  The \textbf{surface integral} of a function $f(x,y,z)$ over a surface
  $S$ in $xyz$ space is given by
  \[
    \iint_S f(\vect{r})\dvar{\sigma}
      =
    \iint_R f(\vect{r}(u,v))|\vect{r}_u\times\vect{r}_v|\dvar{A}
  \]
  where $\vect{r}(u,v)$ is a parameterization from the region $R$ in
  the $uv$ plane to the surface $S$.
\end{definition}

\begin{theorem}
  The surface area of $S$ is given by
  \[
    \iint_S \dvar{\sigma} = \iint_S 1\dvar{\sigma}
  \]
\end{theorem}

          \begin{problem}
            Use the parameterization
            \[
              \vect{r}(\phi,\theta)
                =
              \<
                \sin\phi\cos\theta,
                \sin\phi\sin\theta,
                \cos\phi
              \>
            \]
            from $0\leq\phi\leq\pi,0\leq\theta\leq2\pi$ to the unit
            sphere to show that the surface area of the unit sphere
            is $4\pi$. (Note that this matches the formula $SA=4\pi r^2$ used
            in high school geometry.)
          \end{problem}

          \begin{solution}

          \end{solution}

          \begin{contributors}

          \end{contributors}

          \begin{problem}
            Show that the area of the parallelogram with vertices $(0,0,0)$,
            $(2,1,2)$, $(0,2,-1)$, and $(2,3,1)$ is $3\sqrt{5}$ using a surface
            integral.
            (Hint: use $\vect{r}(u,v)=\<2u,u+2v,2u-v\>$.)
          \end{problem}

          \begin{solution}

          \end{solution}

          \begin{contributors}

          \end{contributors}

\end{document}