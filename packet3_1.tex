\documentclass[letterpaper, twoside, 12pt]{book}
\usepackage{packet}


\begin{document}

\setcounter{chapter}{2}

\chapter{Packet 3.1: Sections 15.1-15.3 and 15.7}

\setcounter{chapter}{15}
\setcounter{section}{0}

\section{Double Integrals over Rectangles} %15.1

\begin{definition}
  We define the \textbf{double integral} of a function $f(x,y)$
  over a region $R$ to be
    \[
      \iint_R f(x,y)\dvar{A} =
      \lim_{n\to\infty}\sum_{i=1}^n f(x_{n,i},y_{n,i})\Delta A_{n,i}
    \]
  where for each positive integer $n$ we've defined a way to
  partition $R$ into $n$ pieces
    \[
      \Delta R_{n,1},\Delta R_{n,2},\dots,\Delta R_{n,n}
    \]
  where $\Delta R_{n,i}$ has area $\Delta A_{n,i}$, contains the
  point $(x_{n,i},y_{n,i})$, and
    \[
      \lim_{n\to\infty} \max(\Delta A_{n,i}) = 0
    \]
\end{definition}

\begin{remark}
  This basically defines the double integral to be the \textbf{Riemann sum}
  of a bunch of rectangular box volumes, just as the single definite integral
  is the Riemann sum of a bunch of rectangle areas. Therefore it represents
  the net volume between the curve $z=f(x,y)$ and the $xy$-plane above/below
  $R$.
\end{remark}

\begin{theorem}
  For the rectangle
  \[
    R: a\leq x\leq b, c\leq y\leq d
  \]
  the \textbf{Midpoint Rule} says that
  \[
    \iint_R f(x,y)\dvar{A}
      \approx
    \sum_{i=1}^m\sum_{j=1}^n f(\overline{x_i},\overline{y_j}) \Delta A
  \]
  where $(\overline{x_i},\overline{y_j})$ is the midpoint of
  the $i\times j$ rectangle.
\end{theorem}

          \begin{problem}
            Divide $R : 0\leq x\leq 4, 0\leq y\leq 2 $
            into four congruent pieces arranged two-by-two,
            and then use the midpoint rule to approximate the double integral
            $\ds\iint_R 2x+2y+4 \dvar{A}$.
          \end{problem}

          \begin{solution}

          \end{solution}

          \begin{contributors}
            % Include the names of everyone in the group who contributed
            % to solving this problem.
          \end{contributors}

          \begin{problem}
            Divide $R : -2\leq x\leq 2, 0\leq y\leq 2 $
            into four congruent pieces arranged two-by-two,
            and then use the midpoint rule to approximate the double integral
            $\ds\iint_R 12x^2y\dvar{A}$
          \end{problem}

          \begin{solution}

          \end{solution}

          \begin{contributors}
            % Include the names of everyone in the group who contributed
            % to solving this problem.
          \end{contributors}

          \begin{problem}
            Divide $R : 0\leq x\leq \pi/2, 0\leq y\leq \pi/2 $
            into four congruent pieces arranged two-by-two,
            and then use the midpoint rule to approximate the double integral
            $\ds\iint_R \cos(x+y)\dvar{A}$
          \end{problem}

          \begin{solution}

          \end{solution}

          \begin{contributors}
            % Include the names of everyone in the group who contributed
            % to solving this problem.
          \end{contributors}




\section{Iterated Integrals} %15.2

\begin{definition}
  If a solid is embedded in $xyz$ space, and $A(x)$ is the area of that
  solid's cross-section for each $x$-value, then the solid's volume is
    \[
      V  = \int_a^b A(x)\dvar{x}
    \]
\end{definition}

\begin{theorem}
  A double integral over a rectangle
    \[
      R: a\leq x\leq b, c\leq y\leq d
    \]
      can be evaluated using the \textbf{iterated integrals}:
    \[
      \iint_R f(x,y)\dvar{A}
        =
      \int_{x=a}^{x=b}\left[\int_{y=c}^{y=d} f(x,y)\dvar{y}\right]\dvar{x}
        =
      \int_{y=c}^{y=d}\left[\int_{x=a}^{x=b} f(x,y)\dvar{x}\right]\dvar{y}
    \]
\end{theorem}

\begin{remark}
  Iterated integrals are often shortened as follows:
    \[
      \int_{a}^{b}\int_{c}^{d} f(x,y)\dvar{y}\dvar{x}
        =
      \int_{x=a}^{x=b}\left[\int_{y=c}^{y=d} f(x,y)\dvar{y}\right]\dvar{x}
    \]
    \[
      \int_{c}^{d}\int_{a}^{b} f(x,y)\dvar{x}\dvar{y}
        =
      \int_{y=c}^{y=d}\left[\int_{x=a}^{x=b} f(x,y)\dvar{x}\right]\dvar{y}
    \]
\end{remark}

\begin{remark}
  When evaluating iterated integrals, only the innermost $d$-variable
  acts as a variable, while other variables act as constants.
  Put another way, find the partial anti-derivatives.
\end{remark}

\begin{remark}
  The order of a double iterated integral
  with constant bounds may
  be reversed by swapping \textbf{both}
  the bounds of
  integration and the differentials $dx$/$dy$.
  (This will not work if there are any variables
  in the bounds as we'll see in the next
  section.)
\end{remark}

          \begin{problem}
            Evaluate $\ds\int_0^3\int_2^4 xy^2 + x^3 \dvar{x}\dvar{y}$.
          \end{problem}

          \begin{solution}

          \end{solution}

          \begin{contributors}
            % Include the names of everyone in the group who contributed
            % to solving this problem.
          \end{contributors}

          \begin{problem}
            If $R : 0\leq x\leq 4, 0\leq y\leq 2 $, then write
            $\ds\iint_R 2x+2y+4 \dvar{A}$
            as an iterated integral. Then evaluate it, comparing its
            value to the approximation you found in the previous section.
          \end{problem}

          \begin{solution}

          \end{solution}

          \begin{contributors}
            % Include the names of everyone in the group who contributed
            % to solving this problem.
          \end{contributors}

          \begin{problem}
            If $R : -2\leq x\leq 2, 0\leq y\leq 2 $, then write
            $\ds\iint_R 12x^2y\dvar{A}$
            as an iterated integral. Then evaluate it, comparing its
            value to the approximation you found in the previous section.
          \end{problem}

          \begin{solution}

          \end{solution}

          \begin{contributors}
            % Include the names of everyone in the group who contributed
            % to solving this problem.
          \end{contributors}

          \begin{problem}
            If $R : 0\leq x\leq \pi/2, 0\leq y\leq \pi/2 $, then write
            $\ds\iint_R \cos(x+y)\dvar{A}$
            as an iterated integral. Then evaluate it, comparing its
            value to the approximation you found in the previous section.
          \end{problem}

          \begin{solution}

          \end{solution}

          \begin{contributors}
            % Include the names of everyone in the group who contributed
            % to solving this problem.
          \end{contributors}



\section{Double Integrals over General Regions} %15.3

\begin{theorem}
  A double integral over a \textbf{Type I} region
  bounded by top and bottom curves
    \[
      R: a\leq x\leq b, g(x)\leq y\leq h(x)
    \]
      can be evaluated using the iterated integral:
    \[
      \iint_R f(x,y)\dvar{A}
        =
      \int_{x=a}^{x=b}\left[\int_{y=g(x)}^{y=h(x)} f(x,y)\dvar{y}\right]\dvar{x}
    \]
\end{theorem}

\begin{theorem}
  A double integral over a \textbf{Type II} region
  bounded by right and left curves
    \[
      R: g(y)\leq x\leq h(y), c\leq y\leq d
    \]
      can be evaluated using the iterated integral:
    \[
      \iint_R f(x,y)\dvar{A}
        =
      \int_{y=c}^{y=d}\left[\int_{x=g(y)}^{x=h(y)} f(x,y)\dvar{x}\right]\dvar{y}
    \]
\end{theorem}

\begin{remark}
  Note that you \textit{never} have variables of integration on the
  outside-most integral in an iterated integral.
\end{remark}

          \begin{problem}
            Evaluate $\ds\int_0^4\int_{\sqrt{y}}^2 6x+30y\dvar{x}\dvar{y}$.
          \end{problem}

          \begin{solution}

          \end{solution}

          \begin{contributors}

          \end{contributors}

          \begin{problem}
            Evaluate $\iint_R 6xy+3 \dvar{A}$
            where $R$ is the region between $x=4-y^2$ and $x=y^2-4$.
          \end{problem}

          \begin{solution}

          \end{solution}

          \begin{contributors}

          \end{contributors}

          \begin{problem}
            Evaluate $\iint_R 6xy+3 \dvar{A}$
            where $R$ is the region between $x=4-y^2$ and $x=y^2-4$.
          \end{problem}

          \begin{solution}

          \end{solution}

          \begin{contributors}

          \end{contributors}

          \begin{problem}
            Evaluate $\iint_R 1 \dvar{A}$
            where $R$ is the triangle with vertices $(0,0)$,
            $(1,1)$, and $(1,2)$.
          \end{problem}

          \begin{solution}

          \end{solution}

          \begin{contributors}

          \end{contributors}

\begin{remark}
  You cannot blindly switch the bounds of integration to change
  the order of integration for a non-rectangular
  region. However, if the region is both Type I and Type II, then the
  order of integration may be swapped by reinterpreting the region
  as the opposite type.
\end{remark}

          \begin{problem}
            Evaluate the Type I iterated integral
            $\ds\int_0^1\int_x^1 \frac{2}{\sqrt{4+y^2}}\dvar{y}\dvar{x}$
            by first rewriting it as a Type II iterated integral.
          \end{problem}

          \begin{solution}

          \end{solution}

          \begin{contributors}

          \end{contributors}

          \begin{problem}
            Evaluate the Type II iterated integral
            $\ds\int_0^1\int_{\sqrt{y}}^1 3\pi \sin(\pi x^3)\dvar{x}\dvar{y}$
            by first rewriting it as a Type I iterated integral.
          \end{problem}

          \begin{solution}

          \end{solution}

          \begin{contributors}

          \end{contributors}

\begin{theorem}
  The area of a region $R$ in the plane is
  \[A = \iint\limits_R\,dA=\iint\limits_R 1\dvar A\]
\end{theorem}

          \begin{problem}
            Express the area of the parallelogram with vertices
            $(-1,2)$, $(3,2)$, $(4,1)$, $(0,1)$
            as a double iterated integral.
          \end{problem}

          \begin{solution}

          \end{solution}

          \begin{contributors}

          \end{contributors}

\begin{definition}
  The average value of a two-variable function $f$ over a region $R$ is
  \[
    \frac{1}{\text{Area of }R}\iint_R f(x,y)\dvar{A}
  \]
\end{definition}

          \begin{problem}
            Express the average value of $f(x,y)=\sin(\frac{x}{2y})$ over
            the triangle with vertices $(0,1)$, $(1,1)$, $(0,2)$
            as a double iterated integral.
          \end{problem}

          \begin{solution}

          \end{solution}

          \begin{contributors}

          \end{contributors}

\begin{definition}
  The \textbf{centroid} $(\overline{x},\overline{y})$ of a region $R$ is the
  average position of all the points in $R$.
\end{definition}

          \begin{problem}
            Prove that the centroid
            of a region $R$ is given by the expressions:
              \[
                \overline{x}
                  =
                \frac{
                  1
                }{
                  \iint_R 1\dvar{R}
                }
                  \iint_R x\dvar{R}
              \]
              \[
                \overline{y}
                  =
                \frac{
                  1
                }{
                  \iint_R 1\dvar{R}
                }
                  \iint_R y\dvar{R}
              \]
            (It's okay to prove one of these and say the other follows from
            basically the same argument.)
          \end{problem}

          \begin{solution}

          \end{solution}

          \begin{contributors}

          \end{contributors}

\begin{remark}
  Not every region is Type I or Type II.
\end{remark}

\begin{theorem}
  If $R$ can be split into two regions $R_1,R_2$, then
  \[
    \iint\limits_R f(x,y)\dvar A
      =
    \iint\limits_{R_1} f(x,y)\dvar A + \iint\limits_{R_2} f(x,y)\dvar A
  \]
\end{theorem}

          \begin{problem}
            Express $\iint_R xe^{x+y}\dvar{A}$ as the sum of two
            iterated integrals, where $R$ is the quadrilateral with
            vertices at $(0,0)$, $(1,1)$, $(2,0)$, and $(1,2)$.
          \end{problem}

          \begin{solution}

          \end{solution}

          \begin{contributors}

          \end{contributors}




\setcounter{section}{6}
\section{Triple Integrals} %15.7

\begin{definition}
  The \textbf{triple integral} of a function $f(x,y,z)$ over a solid $D$ is
  given by
    \[
      \iiint_D f(x,y,z)\dvar{V} =
      \lim_{n\to\infty}\sum_{i=1}^n f(x_{n,i},y_{n,i},z_{n,i})\Delta V_{n,i}
    \]
  where for each positive integer $n$ we've defined a way to partition
  $D$ into $n$ pieces
    \[
      \Delta D_{n,1},\Delta D_{n,2},\dots,\Delta D_{n,n}
    \]
  where $\Delta D_{n,i}$ has volume $\Delta V_{n,i}$,
  contains the point $(x_{n,i},y_{n,i},z_{n,i})$,
  and \[\lim_{n\to\infty} \max(\Delta V_{n,i}) = 0\]
\end{definition}

\begin{theorem}
  The triple integral over the rectangular box
    \[
      D: a_1\leq x\leq a_2, b_1\leq y\leq b_2, c_1\leq z\leq c_2
    \]
  can be expressed as the iterated integrals:
    \[
      \iiint_D f(x,y,z)\dvar{V}
        =
      \int_{a_1}^{a_2}\int_{b_1}^{b_2}\int_{c_1}^{c_2}
      f(x,y,z)\dvar{z}\dvar{y}\dvar{x}
    \]
    \[
      = \int_{b_1}^{b_2}\int_{c_1}^{c_2}\int_{a_1}^{a_2}
      f(x,y,z)\dvar{x}\dvar{z}\dvar{y}
      = \int_{a_1}^{a_2}\int_{c_1}^{c_2}\int_{b_1}^{b_2}
      f(x,y,z)\dvar{y}\dvar{z}\dvar{x}
      = \cdots
    \]
\end{theorem}

          \begin{problem}
            Evaluate $\ds\iiint_D 8xz-y^2\dvar{V}$ where
            $D$ is the unit cube: $0\leq x\leq 1$,
            $0\leq y\leq 1$, $0\leq z\leq 1$.
          \end{problem}

          \begin{solution}

          \end{solution}

          \begin{contributors}

          \end{contributors}

\begin{theorem}
  If the solid $D$ is bounded by the surfaces
    \[
      h_1(x,y)\leq z\leq h_2(x,y)
    \]
  and has shadow $R$ in the $xy$-plane, then
    \[
      \iiint_D f(x,y,z)\dvar{V}
        =
      \iint_R\left[ \int_{h_1(x,y)}^{h_2(x,y)} f(x,y,z) \dvar{z} \right]\dvar{A}
    \]
\end{theorem}

\begin{remark}
  $z$ may be replaced with $x$ or $y$ by changing the orientation to
  let $x$ or $y$ be ``up''.
\end{remark}

          \begin{problem}
            Evaluate
            $\ds\int_{-1}^1\int_{1+y}^{2+y}\int_0^2 z \dvar{x}\dvar{z}\dvar{y}$.
          \end{problem}

          \begin{solution}

          \end{solution}

          \begin{contributors}

          \end{contributors}

          \begin{problem}
            Express $\ds\iiint_D xy^2z \dvar{V}$ as a triple iterated integral,
            where $D$ is the solid in the first octant bounded by the
            coordinate planes, $z=1-y^2$, and $x=4$.
          \end{problem}

          \begin{solution}

          \end{solution}

          \begin{contributors}

          \end{contributors}

\begin{theorem}
  The volume of a solid $D$ in $xyz$ space is
  \[V = \iiint\limits_D\dvar{V}=\iiint\limits_D 1 \dvar{V}\]
\end{theorem}

          \begin{problem}
            Express the volume of the pyramid with vertices
            $(0,0,0)$, $(3,0,0)$, $(0,2,0)$, and $(0,0,1)$
            as a triple iterated integral.
          \end{problem}

          \begin{solution}

          \end{solution}

          \begin{contributors}

          \end{contributors}

\begin{definition}
  The average value of a three-variable function $f$ over a solid $D$ is
  \[
    \frac{1}{\text{Volume of }D}\iiint_D f(x,y,z)\dvar{V}
  \]
\end{definition}

          \begin{problem}
            Express the average value of the function $f(x,y,z)=z+xy$
            over the solid bounded by the surfaces
            $z=4-x^2-y^2$ and $z=4x^2+4y^2-16$.
          \end{problem}

          \begin{solution}

          \end{solution}

          \begin{contributors}

          \end{contributors}

\begin{theorem}
  If $D$ can be split into two solids $D_1,D_2$, then
  \[
    \iiint\limits_D f(x,y,z)\dvar V
      =
    \iiint\limits_{D_1} f(x,y,z)\dvar V + \iiint\limits_{D_2} f(x,y,z)\dvar V
  \]
\end{theorem}

\end{document}