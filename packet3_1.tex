\documentclass[letterpaper, twoside, 12pt]{book}
\usepackage{packet}


\begin{document}

\setcounter{chapter}{2}

\chapter{Packet 3.1: Sections 15.1-15.3 and 15.7}

\setcounter{chapter}{15}
\setcounter{section}{0}

\section{Double Integrals over Rectangles} %15.1

\begin{definition}
  We define the \textbf{double integral} of a function $f(x,y)$
  over a region $R$ to be
    \[
      \iint_R f(x,y)\dvar{A} =
      \lim_{n\to\infty}\sum_{i=1}^n f(x_{n,i},y_{n,i})\Delta A_{n,i}
    \]
  where for each positive integer $n$ we've defined a way to
  partition $R$ into $n$ pieces
    \[
      \Delta R_{n,1},\Delta R_{n,2},\dots,\Delta R_{n,n}
    \]
  where $\Delta R_{n,i}$ has area $\Delta A_{n,i}$, contains the
  point $(x_{n,i},y_{n,i})$, and
    \[
      \lim_{n\to\infty} \max(\Delta A_{n,i}) = 0
    \]
\end{definition}

\begin{remark}
  This basically defines the double integral to be the \textbf{Riemann sum}
  of a bunch of rectangular box volumes, just as the single definite integral
  is the Riemann sum of a bunch of rectangle areas. Therefore it represents
  the net volume between the curve $z=f(x,y)$ and the $xy$-plane above/below
  $R$.
\end{remark}

\begin{theorem}
  For the rectangle
  \[
    R: a\leq x\leq b, c\leq y\leq d
  \]
  the \textbf{Midpoint Rule} says that
  \[
    \iint_R f(x,y)\dvar{A}
      \approx
    \sum_{i=1}^m\sum_{j=1}^n f(\overline{x_i},\overline{y_j}) \Delta A
  \]
  where $(\overline{x_i},\overline{y_j})$ is the midpoint of
  the $i\times j$ rectangle.
\end{theorem}

          \begin{problem}
            Divide $R : 0\leq x\leq 4, 0\leq y\leq 2 $
            into four congruent pieces arranged two-by-two,
            and then use the midpoint rule to approximate the double integral
            $\ds\iint_R 2x+2y+4 \dvar{A}$.
          \end{problem}

          \begin{solution}

          \end{solution}

          \begin{contributors}
            % Include the names of everyone in the group who contributed
            % to solving this problem.
          \end{contributors}

          \begin{problem}
            Divide $R : -2\leq x\leq 2, 0\leq y\leq 2 $
            into four congruent pieces arranged two-by-two,
            and then use the midpoint rule to approximate the double integral
            $\ds\iint_R 12x^2y\dvar{A}$
          \end{problem}

          \begin{solution}

          \end{solution}

          \begin{contributors}
            % Include the names of everyone in the group who contributed
            % to solving this problem.
          \end{contributors}

          \begin{problem}
            Divide $R : 0\leq x\leq \pi/2, 0\leq y\leq \pi/2 $
            into four congruent pieces arranged two-by-two,
            and then use the midpoint rule to approximate the double integral
            $\ds\iint_R \cos(x+y)\dvar{A}$
          \end{problem}

          \begin{solution}

          \end{solution}

          \begin{contributors}
            % Include the names of everyone in the group who contributed
            % to solving this problem.
          \end{contributors}




\section{Iterated Integrals} %15.2

\begin{definition}
  If a solid is embedded in $xyz$ space, and $A(x)$ is the area of that
  solid's cross-section for each $x$-value, then the solid's volume is
    \[
      V  = \int_a^b A(x)\dvar{x}
    \]
\end{definition}

\begin{theorem}
  A double integral over a rectangle
    \[
      R: a\leq x\leq b, c\leq y\leq d
    \]
      can be evaluated using the \textbf{iterated integrals}:
    \[
      \iint_R f(x,y)\dvar{A}
        =
      \int_{x=a}^{x=b}\left[\int_{y=c}^{y=d} f(x,y)\dvar{y}\right]\dvar{x}
        =
      \int_{y=c}^{y=d}\left[\int_{x=a}^{x=b} f(x,y)\dvar{x}\right]\dvar{y}
    \]
\end{theorem}

\begin{remark}
  Iterated integrals are often shortened as follows:
    \[
      \int_{a}^{b}\int_{c}^{d} f(x,y)\dvar{y}\dvar{x}
        =
      \int_{x=a}^{x=b}\left[\int_{y=c}^{y=d} f(x,y)\dvar{y}\right]\dvar{x}
    \]
    \[
      \int_{c}^{d}\int_{a}^{b} f(x,y)\dvar{x}\dvar{y}
        =
      \int_{y=c}^{y=d}\left[\int_{x=a}^{x=b} f(x,y)\dvar{x}\right]\dvar{y}
    \]
\end{remark}

\begin{remark}
  When evaluating iterated integrals, only the innermost $d$-variable
  acts as a variable, while other variables act as constants.
  Put another way, find the partial anti-derivatives.
\end{remark}

\begin{remark}
  The order of a double iterated integral
  with constant bounds may
  be reversed by swapping \textbf{both}
  the bounds of
  integration and the differentials $dx$/$dy$.
  (This will not work if there are any variables
  in the bounds as we'll see in the next
  section.)
\end{remark}

          \begin{problem}
            Evaluate $\ds\int_0^3\int_2^4 xy^2 + x^3 \dvar{x}\dvar{y}$.
          \end{problem}

          \begin{solution}

          \end{solution}

          \begin{contributors}
            % Include the names of everyone in the group who contributed
            % to solving this problem.
          \end{contributors}

          \begin{problem}
            If $R : 0\leq x\leq 4, 0\leq y\leq 2 $, then write
            $\ds\iint_R 2x+2y+4 \dvar{A}$
            as an iterated integral. Then evaluate it, comparing its
            value to the approximation you found in the previous section.
          \end{problem}

          \begin{solution}

          \end{solution}

          \begin{contributors}
            % Include the names of everyone in the group who contributed
            % to solving this problem.
          \end{contributors}

          \begin{problem}
            If $R : -2\leq x\leq 2, 0\leq y\leq 2 $, then write
            $\ds\iint_R 12x^2y\dvar{A}$
            as an iterated integral. Then evaluate it, comparing its
            value to the approximation you found in the previous section.
          \end{problem}

          \begin{solution}

          \end{solution}

          \begin{contributors}
            % Include the names of everyone in the group who contributed
            % to solving this problem.
          \end{contributors}

          \begin{problem}
            If $R : 0\leq x\leq \pi/2, 0\leq y\leq \pi/2 $, then write
            $\ds\iint_R \cos(x+y)\dvar{A}$
            as an iterated integral. Then evaluate it, comparing its
            value to the approximation you found in the previous section.
          \end{problem}

          \begin{solution}

          \end{solution}

          \begin{contributors}
            % Include the names of everyone in the group who contributed
            % to solving this problem.
          \end{contributors}

\end{document}