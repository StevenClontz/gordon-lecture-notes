\documentclass[letterpaper, twoside, 12pt]{book}
\usepackage{packet}


\begin{document}

\setcounter{chapter}{10}
\setcounter{section}{3}


\section{The Cross Product}

\begin{definition}
  For any two non-parallel and non-zero vectors $\harpvec{u}$, $\harpvec{v}$
  in $\mathbb R^3$,
  the \textbf{Right-Hand Rule} gives a specific direction orthogonal to both:
  position both vectors at the origin, and draw a line orthogonal to the
  plane containing both vectors. Then place your right thumb near
  $\harpvec{u}$ and your right index finger near $\harpvec{v}$.
  The direction on the orthogonal line given by extending your middle
  finger is the direction given by the RHR.
\end{definition}

\begin{definition}
  Let \(\harpvec u,\harpvec v\) be vectors.
  Their \textbf{cross product} $\harpvec{u}\times\harpvec{v}$ is the vector
  constructed as follows:
  \begin{enumerate}
    \item If either of \(\harpvec u,\harpvec v\) is the zero vector
      \(\harpvec 0\), then \(\harpvec u\times\harpvec v=\harpvec 0\).
    \item If \(\harpvec u,\harpvec v\) are parallel,
      then \(\harpvec u\times\harpvec v=\harpvec 0\).
    \item Otherwise, let \(\harpvec n\) be the unit vector given by the
     vectors \(\harpvec u,\harpvec v\) and the RHR, and let \(a\) be the area
     of the parallelogram determined by the vectors \(\harpvec u,\harpvec v\).
     Then \(\harpvec u\times\harpvec v=a\harpvec n\).
  \end{enumerate}
  % Let $\theta$ be the angle between two non-zero vectors $\harpvec{u}$,
  % $\harpvec{v}$ in $\mathbb{R}^3$, and let $\harpvec{n}$ be the direction
  % given by the Right-Hand Rule.

  %  to both which follows the Right-Hand Rule and has magnitude
  % equal to the area of the parallelogram formed from both.
  % \[
  %   \harpvec{u}\times\harpvec{v}
  %     =
  %   (|\harpvec{u}||\harpvec{v}|\sin\theta)\harpvec{n}
  % \]
  % \[
  %   |\harpvec{u}\times\harpvec{v}|
  %     =
  %   |\harpvec{u}||\harpvec{v}|\sin\theta
  % \]
\end{definition}

\begin{theorem}
  The cross products of the standard unit vectors are given as follows:
  \begin{itemize}
    \item $\veci \times \vecj = \veck$
    \item $\vecj \times \veci = -\veck$
    \item $\vecj \times \veck = \veci$
    \item $\veck \times \vecj = -\veci$
    \item $\veck \times \veci = \vecj$
    \item $\veci \times \veck = -\vecj$
    \item $\veci \times \veci = \harpvec 0$
    \item $\vecj \times \vecj = \harpvec 0$
    \item $\veck \times \veck = \harpvec 0$
  \end{itemize}
\end{theorem}

\begin{theorem}
  The following properties hold for any three vectors $\harpvec{u}$, $\harpvec{v}$,
  $\harpvec{w}$ and scalars $a$,$b$.
  \begin{itemize}
  \item $\harpvec{v} \times \harpvec{u} = -(\harpvec{u} \times \harpvec{v})$
  \item $(a\harpvec{u}) \times (b\harpvec{v}) = (ab)(\harpvec{u} \times \harpvec{v})$
  \item
    $\harpvec{u} \times (\harpvec{v} + \harpvec{w}) =
    \harpvec{u} \times \harpvec{v} + \harpvec{u} \times \harpvec{w}$
  \item
    $(\harpvec{v} + \harpvec{w}) \times \harpvec{u} =
    \harpvec{v} \times \harpvec{u} + \harpvec{w} \times \harpvec{u}$
  \end{itemize}
\end{theorem}

\begin{problem}
  Compute \((3\veci-4\vecj)\times(\vecj+2\veck)\).
\end{problem}

\begin{definition}
  A \textbf{determinant} is shorthand for writing the following
  algebraic expressions:
    \[
      \begin{array}{|c c|}
      a_1 & a_2 \\
      b_1 & b_2 \\
      \end{array}
        =
      a_1b_2 - a_2b_1
    \]
    \[
      \begin{array}{|c c c|}
      a_1 & a_2 & a_3 \\
      b_1 & b_2 & b_3 \\
      c_1 & c_2 & c_3 \\
      \end{array}
        =
      a_1 \,
      \begin{array}{|c c|}
      b_2 & b_3 \\
      c_2 & c_3 \\
      \end{array}
        -
      a_2 \,
      \begin{array}{|c c|}
      b_1 & b_3 \\
      c_1 & c_3 \\
      \end{array}
        +
      a_3 \,
      \begin{array}{|c c|}
      b_1 & b_2 \\
      c_1 & c_2 \\
      \end{array}
    \]
\end{definition}

\begin{theorem}
  The area of a parallelogram determined by two
  vectors \(\harpvec u,\harpvec v\) with angle \(\theta\) is given by
  \(
    \|\harpvec u\|\|\harpvec v\|\sin\theta
  \).
\end{theorem}

\begin{problem}
  Find the area of the parallelogram determined by the vectors
  \(\<0,3\>\) and \(\<2,2\>\)
\end{problem}

\begin{theorem}
  The area of a parallelogram determined by two \(2D\)
  vectors \(\harpvec u,\harpvec v\) with angle \(\theta\) is given by
  the absolute value of the determinant
  \(
    \begin{array}{|c c|}
    u_1 & u_2 \\
    v_1 & v_2 \\
    \end{array}
  \).
\end{theorem}

\begin{problem}
  Use this to resolve the previous problem.
\end{problem}

\begin{problem}
  Find the area of the triangle with vertices at \((2,3)\), \((-1,4)\),
  and \((1,1)\)
\end{problem}

\begin{theorem}
  The volume of a parallelepiped determined by three three-dimensional
  vectors \(\harpvec u,\harpvec v,\harpvec w\) is given by the absolute
  value of their
  \textbf{triple scalar product}, the determinant
  \(
    \begin{array}{|c c c|}
    u_1 & u_2 & u_3 \\
    v_1 & v_2 & v_3 \\
    w_1 & w_2 & w_3 \\
    \end{array}
  \).
\end{theorem}

\begin{problem}
  Find the volume of the parallelepiped determined by the vectors
  \(\<1,2,3\>\), \(\<0,-1,4\>\), and \(\<2,2,0\>\).
\end{problem}



\begin{theorem}
  By breaking up $\harpvec{u}$, $\harpvec{v}$ into standard unit vectors:
  \[
  \harpvec{u} \times \harpvec{v}
    =
  \begin{array}{|c c c|}
  \veci & \vecj & \veck \\
  u_1 & u_2 & u_3 \\
  v_1 & v_2 & v_3 \\
  \end{array}
    =
  \begin{array}{|c c|}
  u_2 & u_3 \\
  v_2 & v_3 \\
  \end{array}
  ~\veci-
  \begin{array}{|c c|}
  u_1 & u_3 \\
  v_1 & v_3 \\
  \end{array}
  ~\vecj+
  \begin{array}{|c c|}
  u_1 & u_2 \\
  v_1 & v_2 \\
  \end{array}
  ~\veck
  \]
\end{theorem}

\begin{problem}
  Recompute \((3\veci-4\vecj)\times(\vecj+2\veck)\).
\end{problem}

\begin{problem}
  Find the area of the parallelogram determined by $\harpvec{u}=\<4,-3,0\>$
  and $\harpvec{v}=\<2,6,-3\>$.
\end{problem}

\begin{problem}
  Find a unit vector orthogonal to both $\harpvec{u}=\<4,-3,0\>$
  and $\harpvec{v}=\<2,6,-3\>$.
\end{problem}

\begin{theorem}
  The triple scalar product of three vectors is also given by
    \[
      \harpvec{w}\cdot(\harpvec{u}\times\harpvec{v}) =
      (\harpvec{u}\times\harpvec{v})\cdot\harpvec{w} =
      \begin{array}{|c c c|}
      u_1 & u_2 & u_3 \\
      v_1 & v_2 & v_3 \\
      w_1 & w_2 & w_3 \\
      \end{array}
    \]
\end{theorem}


\begin{definition}
  The torque $\tau$ done by a force vector $\harpvec{F}$ on an arm given by
  $\harpvec{D}$ is given by
  \[
    \tau = |\harpvec{F} \times \harpvec{D}|
      =
    |\harpvec{F}||\harpvec{D}|\sin \theta
  \]
\end{definition}

\begin{problem}
  Find the torque enacted by the force \(\<2,2,-2\>\) on a wrench at the point
  \((4,3,2)\) and bolt centered at the point \((1,0,-2)\).
\end{problem}


\end{document}