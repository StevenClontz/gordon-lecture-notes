\documentclass[letterpaper, twoside, 12pt]{book}
\usepackage{packet}


\begin{document}

\setcounter{chapter}{1}

\chapter{Part 2.3: Sections 14.4-14.6}

\setcounter{chapter}{14}
\setcounter{section}{3}

\section{Tangent Planes and Linear Approximations} %14.4

\begin{definition}
  A \textbf{normal vector} to a surface is a vector normal to
  any vector tangent to a curve on the surface.
\end{definition}

\begin{theorem}
  Let $f(x,y)$ be a function of two variables with continuous partial
  derivatives, and let $(a,b)$ be a point in the interior of $f$'s
  domain. Then $\<f_x(a,b),f_y(a,b),-1\>$ is normal to the surface
  at the point $(a,b,f(a,b))$.
\end{theorem}

          \begin{problem}
            OPTIONAL. Prove the previous theorem by using the curves
            $\vect{r}(t)=\<a,t,f(a,t)\>$ and
            $\vect{q}(t)=\<t,b,f(t,b)\>$ to yield the tangent vectors
            $\<0,1,f_y(a,b)\>$ and $\<1,0,f_x(a,b)\>$.
          \end{problem}

          \begin{solution}

          \end{solution}

\begin{definition}
  The \textbf{tangent plane} to a surface at a point is the plane passing
  through that point sharing the same normal vectors as the surface.
\end{definition}

\begin{theorem}
  The tangent plane to the surface $z=f(x,y)$ above the point $(a,b)$
  is given by the equation
    \[
      z = f(a,b) + f_x(a,b)(x-a)+f_y(a,b)(y-b)
    \]
\end{theorem}

          \begin{problem}
            Prove the previous theorem.
          \end{problem}

          \begin{solution}

          \end{solution}

          \begin{problem}
            Find an equation for the plane tangent to the surface
            $z=4x^2+y^2$ above the point $(1,-1)$.
          \end{problem}

          \begin{solution}

          \end{solution}

\begin{definition}
  The \textbf{linearization} $L(x,y)$ of a function $f(x,y)$
  at the point $(a,b)$ is given by the formula:
  \[
    L(x,y) = f(a,b)+f_x(a,b)(x-a)+f_y(a,b)(y-b)
  \]
\end{definition}

\begin{definition}
  A function $f$ is \textbf{differentiable} at a point if its linearization
  at that point approximates the value of the function nearby.
\end{definition}

\begin{remark}
  Basically, a differentiable function is one which looks similar to
  its tangent planes when zoomed in sufficiently far.
\end{remark}

          \begin{problem}
            Approximate the value of the differentiable function
            $f(x,y)=4xy+3y^2$ at $(1.1,-2.05)$ by using its linearization
            at the point $(1,-2)$.
            Then use a calculator to approximate $f(1.1,-2.05)$.
          \end{problem}

          \begin{solution}

          \end{solution}


\section{The Chain Rule}%14.5

\begin{definition}
  The \textbf{gradient} of a multi-variable function is the vector containing
  all its partial derivatives:
    \[
      \nabla f = \<f_x,f_y\>
    \]
    \[
      \nabla g = \<g_x,g_y,g_z\>
    \]
\end{definition}

          \begin{problem}
            Compute the gradient of the function $f(x,y,z)=4x\cos z-y^2$.
            Then compute its value at the point $(1,-2,0)$.
          \end{problem}

          \begin{solution}

          \end{solution}

\begin{remark}
  If $f(P)$ is a function of multiple variables, and
  $\vect{r}(t)$ is a vector function of $t$, then
  $f(\vect{r}(t))$ is a function of $t$.
\end{remark}

          \begin{problem}
            Let $f(x,y)=x^2y+3y^2$ and
            $\vect{r}(t)=\<x(t),y(t)\>=\<t+1,\sqrt{t}\>$.
            Write $f(\vect{r}(t))$ in terms of $t$ only, then
            compute $\frac{df}{dt}$.
          \end{problem}

          \begin{solution}

          \end{solution}

\begin{remark}
  The chain rule for single-variable functions may be written as
  \[
    \frac{d}{dx}\left[f(u(x))\right]
      =
    \frac{df}{dx}
      =
    \frac{df}{du}
    \frac{du}{dx}
      =
    f'(u(x))
    u'(x)
  \]
\end{remark}

\begin{theorem}
  Let $f(P)$ be a function of multiple variables and $\vect{r}(t)$
  be a function of $t$. Then the derivative of $f$ with respect to $t$
  may be computed using the \textbf{Chain Rule}:
  \[
    \frac{d}{dt}\left[f(\vect{r}(t))\right]
      =
    \frac{df}{dt}
      =
    \nabla f
    \cdot
    \frac{d\vect{r}}{dt}
  \]
\end{theorem}

          \begin{problem}
            Let $f$ and $\vect{r}$ be defined as in the previous problem.
            Use the Chain Rule to compute $\frac{df}{dt}$.
          \end{problem}

          \begin{solution}

          \end{solution}

          \begin{problem}
            Let $f(x,y,z)=xyz^2$, $x(t)=2t+1$, $y(t)=t^2+1$,
            and $z(t)=1-t^3$.
            Compute $\frac{df}{dt}$ at $t=1$.
          \end{problem}

          \begin{solution}

          \end{solution}

\begin{theorem}
  Suppose $f(x,y)=c$ defines $y$ as a function of $x$. Then
  \[
    \frac{dy}{dx} = -\frac{f_x}{f_y}
  \]
\end{theorem}

          \begin{problem}
            Prove the previous theorem. (Part of the solution has
            been provided for you.)
          \end{problem}

          \begin{solution}
            Let $y(x)$ be the function defined by $f(x,y(x))=c$, and
            then let $t=x$. It follows that
            $f(t,y(t))=f(\vect{r}(t))=c$, so by the Chain Rule,
            \[
              \frac{d}{dt}[f(\vect{r}(t))]=\frac{d}{dt}[c]
            \]
            \[
              \dots
            \]
            Since $\frac{dy}{dt}=-\frac{f_x}{f_y}$ and $t=x$, we conclude that
            $\frac{dy}{dx}=-\frac{f_x}{f_y}$.
          \end{solution}

          \begin{problem}
            Find the rate of change $\frac{dy}{dx}$ for
            $xy^2=3x-2y$ at $(-1,3)$.
          \end{problem}

          \begin{solution}

          \end{solution}


\section{Directional Derivatives and the Gradient Vector}%14.6

\begin{definition}
  Let $\vect{u}$ be a direction. The
  \textbf{derivative of $f$ in the direction $\vect u$}, denoted
  $D_{\vect{u}}f$, is given by $\frac{df}{ds}$ where $s$ is the
  arclength parameter for the line oriented in the direction $\vect{u}$.
\end{definition}

\begin{theorem}
  The directional derivative is the dot product of the gradient
  vector and $\vect{u}$:
  \[
    D_{\vect{u}}f = \nabla f \cdot \vect{u}
  \]
\end{theorem}

\begin{remark}
  The proof of the previous theorem follows from the fact that if
  $\vect{r}$ is the line oriented in the direction $\vect{u}$, then
  $\frac{df}{ds}=\nabla f \cdot \frac{d\vect{r}}{ds}$ and
  $\frac{d\vect{r}}{ds}=\vect{u}$.
\end{remark}

          \begin{problem}
            Find the rate of change of $f(x,y,z)=xz^3+3yz$ in the direction
            $\vect{u}=\<\frac{1}{3},-\frac{2}{3},\frac{2}{3}\>$
            at the point $P_0=(-2,0,1)$.
          \end{problem}

          \begin{solution}

          \end{solution}

          \begin{problem}
            Find the rate of change of $f(x,y)=xy^2+3y$
            in the direction of $\vect{A}=\<2,2\>$
            at the point $P_0=(2,0)$. (Note that $\vect{A}$ isn't
            a unit vector, so you'll need to find its direction first.)
          \end{problem}

          \begin{solution}

          \end{solution}

          \begin{problem}
            Show that the rate of change of $f$ in the direction of
            $\vecj$ is same thing as the partial derivative of $f$
            with respect to $y$.
          \end{problem}

          \begin{solution}

          \end{solution}

\end{document}