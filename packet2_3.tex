\documentclass[letterpaper, twoside, 12pt]{book}
\usepackage{packet}


\begin{document}

\setcounter{chapter}{1}

\chapter{Part 2: Sections 14.4-14.6}

\setcounter{chapter}{14}
\setcounter{section}{3}

\section{Tangent Planes and Linear Approximations} %14.4

\begin{definition}
  A \textbf{normal vector} to a surface is a vector normal to
  any vector tangent to a curve on the surface.
\end{definition}

\begin{theorem}
  Let $f(x,y)$ be a function of two variables with continuous partial
  derivatives, and let $(a,b)$ be a point in the interior of $f$'s
  domain. Then $\<f_x(a,b),f_y(a,b),-1\>$ is normal to the surface
  at the point $(a,b,f(a,b))$.
\end{theorem}

          \begin{problem}
            OPTIONAL. Prove the previous theorem by using the curves
            $\vect{r}(t)=\<t,b,f(t,b)\>$ and
            $\vect{q}(t)=\<a,t,f(a,t)\>$ to yield the tangent vectors
            $\<1,0,f_x(a,b)\>$ and $\<0,1,f_y(a,b)\>$.
          \end{problem}

          \begin{solution}

          \end{solution}

\begin{definition}
  The \textbf{tangent plane} to a surface at a point is the plane passing
  through that point sharing the same normal vectors as the surface.
\end{definition}

\begin{theorem}
  The tangent plane to the surface $z=f(x,y)$ above the point $(a,b)$
  is given by the equation
    \[
      z = f(a,b) + f_x(a,b)(x-a)+f_y(a,b)(y-b)
    \]
\end{theorem}

          \begin{problem}
            Prove the previous theorem.
          \end{problem}

          \begin{solution}

          \end{solution}

          \begin{problem}
            Find an equation for the plane tangent to the surface
            $z=4x^2+y^2$ above the point $(1,-1)$.
          \end{problem}

          \begin{solution}

          \end{solution}

\begin{definition}
  The \textbf{linearization} $L(x,y)$ of a function $f(x,y)$
  at the point $(a,b)$ is given by the formula:
  \[
    L(x,y) = f(a,b)+f_x(a,b)(x-a)+f_y(a,b)(y-b)
  \]
\end{definition}

\begin{definition}
  A function $f$ is \textbf{differentiable} at a point if its linearization
  at that point approximates the value of the function nearby.
\end{definition}

\begin{remark}
  Basically, a differentiable function is one which looks similar to
  its tangent planes when zoomed in sufficiently far.
\end{remark}

          \begin{problem}
            Approximate the value of the differentiable function
            $f(x,y)=4xy+3y^2$ at $(1.1,-2.05)$ by using its linearization
            at the point $(1,-2)$.
            Then use a calculator to approximate $f(1.1,-2.05)$.
          \end{problem}

          \begin{solution}

          \end{solution}

\end{document}