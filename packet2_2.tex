\documentclass[letterpaper, twoside, 12pt]{book}
\usepackage{packet}


\begin{document}

\setcounter{chapter}{1}

\chapter{Part 2: Sections 14.1-14.3}

\setcounter{chapter}{14}
\setcounter{section}{0}

\section{Functions of Several Variables} %14.1

\begin{definition}
  A \textbf{function $f$ of two variables} is a rule which assigns a
  real number $f(x,y)$ to each pair of real numbers $(x,y)$ for which
  that rule is defined.
  The collection of such well-defined pairs is called the
  \textbf{domain} $\text{dom}(f)$ of the function, and the set of
  real numbers which
  can possiblely be produced by the function is called its
  \textbf{range} $\text{ran}(f)$.
\end{definition}

\begin{definition}
  The \textbf{level curve} for each $k\in\text{ran}(f)$ is given by the
  equation $f(x,y)=k$.
  The \textbf{graph} of $f$ is a surface in 3D space which visualizes the function, given by the equation $z=f(x,y)$.
\end{definition}

\begin{definition}
  A \textbf{function $f$ of three variables} is a rule which assigns a
  real number $f(x,y,z)$ to each triple of real numbers $(x,y,z)$ for which
  that rule is defined.
  The collection of such well-defined triples is called the
  \textbf{domain} $\text{dom}(f)$ of the function, and the set of
  real numbers which
  can possiblely be produced by the function is called its
  \textbf{range} $\text{ran}(f)$.
\end{definition}

          \begin{problem}
            Let $f(x,y)=x\sin(x+y)$. Give the value of $f(\pi,\frac{\pi}{2})$.
          \end{problem}

          \begin{solution}

          \end{solution}

          \begin{problem}
            Let $f(x,y)=-x-y+2$. In the $xy$-plane, plot the domain of $f$,
            as well as its level curves for $k=-3,0,3$. Then plot the graph
            of $f$ in $xyz$ space.
          \end{problem}

          \begin{solution}

          \end{solution}

          \begin{problem}
            Let $f(x,y)=\sqrt{4-x^2-y^2}$. In the $xy$-plane,
            plot the domain of $f$,
            as well as its level curves for $k=0,\frac{1}{\sqrt{2}},1$.
            Then plot the graph of $f$ in $xyz$ space.
          \end{problem}

          \begin{solution}

          \end{solution}

\begin{definition}
  The \textbf{level surface} for each $k\in\text{ran}(f)$ is given by the
  equation $f(x,y,z)=k$.
  (Since the graph of a three variable function would require four
  variables and therefore is a four-dimensional object, we typically
  don't consider it.)
\end{definition}

          \begin{problem}
            Let $f(x,y,z)=\frac{x+3y^2}{z-2x}$. Give the value of $f(3,-2,1)$.
          \end{problem}

          \begin{solution}

          \end{solution}

          \begin{problem}
            Let $f(x,y,z)=-x^2+y-z^2$. In $xyz$ space, plot the level
            surfaces for $k=-2,0,2$.
          \end{problem}

          \begin{solution}

          \end{solution}

\begin{remark}
  If $P=(x,y)$, then we assume that $f(x,y)=f(P)=f(\vect{P})$.
  If $P=(x,y,z)$, then we assume that $f(x,y,z)=f(P)=f(\vect{P})$.
\end{remark}


\section{Limits and Continuity} %14.2

\begin{definition}
  If the value of the function $f(P)$ becomes arbitrarily close to the number
  $L$ as points $P$ close to $P_0$ are plugged into the function, then the
  \textbf{limit of $f(P)$ as $P$ approaches $P_0$} is $L$:
  \[\lim_{P\to P_0} f(P) = L\]
\end{definition}

\begin{theorem}
  Let $f(x,y)$ be a function of two variables.
  If there exists a curve $y=g(x)$ passing through the
  point $(x_0,y_0)$ such that $\lim_{x\to x_0}f(x,g(x))$ does not exist,
  then $\lim_{(x,y)\to(x_0,y_0)}f(x,y)$ does not exist.
\end{theorem}

          \begin{problem}
            Prove that
              \[
                \lim_{(x,y)\to(0,0)} \frac{x+y}{|x+y|}
              \]
            does not exist by considering the function $g(x)=x$.
          \end{problem}

          \begin{solution}

          \end{solution}

\begin{theorem}
  Let $f(x,y)$ be a function of two variables.
  If there exist curves $y=g(x)$ and $y=h(x)$ passing through the
  point $(x_0,y_0)$ such that
  $\lim_{x\to x_0}f(x,g(x))\not=\lim_{x\to x_0}f(x,h(x))$,
  then $\lim_{(x,y)\to(x_0,y_0)}f(x,y)$ does not exist.
\end{theorem}

          \begin{problem}
            Prove that
              \[
                \lim_{(x,y)\to(0,0)} \frac{x^6+y^2}{x^3y+x^6}
              \]
            does not exist by considering the functions
            $g(x)=x^3$ and $h(x)=2x^3$.
          \end{problem}

          \begin{solution}

          \end{solution}

\begin{theorem}
  The ``Limit Laws'' for single-variable functions also hold for
  multi-variable functions.
    \[
      \lim_{P\to P_0}(f(P)\pm g(P))
        =
      \lim_{P\to P_0}f(P) \pm \lim_{P\to P_0}g(P)
    \]
    \[
      \lim_{P\to P_0}(f(P)\cdot g(P))
        =
      \lim_{P\to P_0}f(P) \cdot \lim_{P\to P_0}g(P)
    \]
    \[
      \lim_{P\to P_0}(kf(P))
        =
      k\lim_{P\to P_0}f(P)
    \]
    \[
      \lim_{P\to P_0}\frac{f(P)}{g(P)}
        =
      \frac{\ds \lim_{P\to P_0}f(P)}{\ds \lim_{P\to P_0}g(P)}
    \]
    \[
      \lim_{P\to P_0}(f(P))^{r/s}
        =
      \left(\lim_{P\to P_0}f(P)\right)^{r/s}
    \]
\end{theorem}

\begin{theorem}
  Let $P_0=(x_0,y_0,z_0)$. Multi-variable limits which only use one
  variable may be reduced to a single-variable limit.
    \[
      \lim_{P\to P_0}f(x) = \lim_{x\to x_0}f(x)
    \]
    \[
      \lim_{P\to P_0}g(y) = \lim_{y\to y_0}g(y)
    \]
    \[
      \lim_{P\to P_0}h(z) = \lim_{z\to z_0}h(z)
    \]
\end{theorem}

          \begin{problem}
            Use the above theorems to rigorously prove that
              \[
                \lim_{(x,y)\to(1,2)}
                \frac{2x+y}{y^2}
                  =
                1
              \]
          \end{problem}

          \begin{solution}

          \end{solution}

\begin{remark}
  Due to the limit laws, the ``just plug it in'' rule applies when
  plugging in does not result in an undefined operation.
\end{remark}

          \begin{problem}
            Compute the limit
              \[
                \lim_{(x,y,z)\to(3,0,-1)}
                \frac{x\cos y}{z+x}
              \]
          \end{problem}

\begin{remark}
  There is no L'Hopital rule for multi-variable limits.
  However, you may still use it once the limit has been reduced
  to a single-variable limit.
\end{remark}

          \begin{problem}
            Compute the limit
              \[
                \lim_{(x,y)\to(3,0)}
                \frac{xy+\sin(2y)}{y}
              \]
          \end{problem}

          \begin{solution}

          \end{solution}

\begin{remark}
  Factoring and canceling (including conjugation tricks) is
  also effective for computing multi-variable limits.
\end{remark}

          \begin{problem}
            Compute the limit
              \[
                \lim_{(x,y,z)\to(1,2,4)}
                \frac{\sqrt{z}-xy}{z-x^2y^2}
              \]
          \end{problem}

          \begin{solution}

          \end{solution}

\begin{definition}
  A function $f(P)$ is \textbf{continuous} if
  $\ds \lim_{P\to P_0}f(P) = f(P_0)$ for all points $P_0$ in its domain.
\end{definition}

\begin{theorem}
  If a multi-variable function is composed of continuous single-variable
  functions, then it is also continuous.
\end{theorem}


\section{Partial Derivatives} %14.3

\begin{definition}
  The \textbf{partial derivative of $f$ with respect to a variable} is
  the rate of change of $f$ as that variable changes and all other variables
  are held constant. For example:
  \[\frac{\p f}{\p x}=f_x(x,y)=\lim_{h\to0}\frac{f(x+h,y)-f(x,y)}{h}\]
  \[\frac{\p g}{\p z}=g_z(x,y,z)=\lim_{h\to0}\frac{g(x,y,z+h)-g(x,y,z)}{h}\]
\end{definition}

          \begin{problem}
            Let $f(x,y,z)=xy^2+2z$.
            Use the definition of a partial derivative to prove that
            $\frac{\p f}{\p y} = 2xy$.
          \end{problem}

          \begin{solution}

          \end{solution}

\begin{theorem}
  Partial derivatives may be computed in the usual way by treating
  all other variables as constants.
\end{theorem}

          \begin{problem}
            Compute both partial derivatives of $f(x,y)=4x^2-5y^3+xy-1$.
          \end{problem}

          \begin{solution}

          \end{solution}

          \begin{problem}
            Compute both partial derivatives of $f(x,y)=\sin(x+3y)$.
          \end{problem}

          \begin{solution}

          \end{solution}

          \begin{problem}
            Compute both partial derivatives of $f(x,y)=e^{xy^2}$.
          \end{problem}

          \begin{solution}

          \end{solution}

\begin{definition}
  \textbf{Second-order partial derivatives} are the result of taking the
  partial derivative of a partial derivative.
\end{definition}

\begin{theorem}
  When computing the second-order partial derivative for a sufficiently
  well-behaved function, the order in which the partial
  derivatives are taken is irrelevant. (This is sometimes called the
  \textbf{Mixed Derivative Theorem}.)
\end{theorem}

          \begin{problem}
            Verify the Mixed Derivative Theorem for
            $f(x,y)=3x^2y^2-x^3+y^4-7$.
          \end{problem}

          \begin{solution}

          \end{solution}

\end{document}