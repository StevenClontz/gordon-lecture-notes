\documentclass[letterpaper, twoside, 12pt]{book}
\usepackage{packet}


\begin{document}

\setcounter{chapter}{3}

\chapter{Packet 4.1: Sections 16.1-16.4}

\setcounter{chapter}{16}
\setcounter{section}{0}

\section{Vector Fields} %16.1

\begin{definition}
  A \textbf{vector field} assigns a vector to each point in 2D or 3D space.
    \[
      \vect{F}=
      \vect{F}(\vect{r})=
      \vect{F}(x,y)=
      \<P(x,y),Q(x,y)\>=
      \<P(\vect{r}),Q(\vect{r})\>=
      \<P,Q\>
    \]
    \[
      \vect{F}=
      \vect{F}(\vect{r})=
      \vect{F}(x,y,z)=
      \<P(x,y,z),Q(x,y,z),R(x,y,z)\>=
      \<P(\vect{r}),Q(\vect{r}),R(\vect{r})\>=
      \<P,Q,R\>
    \]
\end{definition}

          \begin{problem}
            Sketch the vector field $\vect{F}=\<x+y,2y\>$ for
            all $x\in\{0,1,2\}$ and $y\in\{0,1,2\}$.
          \end{problem}

          \begin{solution}

          \end{solution}

          \begin{contributors}

          \end{contributors}

\begin{remark}
  The gradient vector function
    \[
      \nabla f (x,y)
        =
      \<f_x(x,y),f_y(x,y)\>
    \]
    \[
      \nabla f (x,y,z)
        =
      \<f_x(x,y,z),f_y(x,y,z),f_z(x,y,z)\>
    \]
  is a vector field which yields normal vectors
  to the level surfaces of the function $f$.
\end{remark}

          \begin{problem}
            Compute $\nabla f$ for the function
            $f(x,y)=x^2-2xy+y$, and then
            sketch the vector field $\nabla f$
            all $x\in\{0,1,2\}$ and $y\in\{0,1,2\}$.
          \end{problem}

          \begin{solution}

          \end{solution}

          \begin{contributors}

          \end{contributors}


\section{Line Integrals} %16.2

\begin{theorem}
  Some vector functions which parameterize curves follow.
  \begin{itemize}
    \item
    A line segment beginning at $P_0$ and ending at $P_1$:
      \[
        \vect{r}(t) = \vect{P_0} + t\vect{P_0P_1}, 0\leq t\leq 1
      \]
    \item
    A circle centered at the origin with radius $a$:
      \[
        \vect{r}(t) = \<a\cos t,a\sin t\>, 0\leq t\leq 2\pi
        \text{ (full counter-clockwise rotation)}
      \]
      \[
        \vect{r}(t) = \<a\sin t,a\cos t\>, 0\leq t\leq 2\pi
        \text{ (full clockwise rotation)}
      \]
    \item
    A planar curve given by $y=f(x)$ from $(x_0,y_0)$ to $(x_1,y_1)$
      \[
        \vect{r}(t) = \<t,f(t)\>, x_0\leq t\leq x_1
        \text{ (left-to-right)}
      \]
      \[
        \vect{r}(t) = \<-t,f(-t)\>, -x_0\leq t\leq -x_1
        \text{ (right-to-left)}
      \]
    \end{itemize}
\end{theorem}

          \begin{problem}
            Give a vector function which parameterizes the line segment
            from the point $(0,3,-2)$ to the point $(4,-1,0)$.
          \end{problem}

          \begin{solution}

          \end{solution}

          \begin{contributors}

          \end{contributors}

          \begin{problem}
            Give a vector function which parameterizes the curve
            $y=x^3-2x$ from the point $(1,-1)$ to the point $(-1,1)$.
          \end{problem}

          \begin{solution}

          \end{solution}

          \begin{contributors}

          \end{contributors}

          \begin{problem}
            Give a vector function which parameterizes the curve
            $x^2+y^2=9$ from the point $(3,0)$ clockwise to the point $(0,-3)$.
          \end{problem}

          \begin{solution}

          \end{solution}

          \begin{contributors}

          \end{contributors}

\begin{definition}
The \textbf{line integral with respect to arclength} of a function of many
variables $f(\vect{r})$ along a curve $C$ is given by
  \[
    \int_C f(\vect{r})\dvar{s} =
    \lim_{n\to\infty}\sum_{i=1}^n f(\vect{r}_{n,i})\Delta s_{n,i}
  \]
where for each positive integer $n$ we've defined a way to partition $C$
into $n$ pieces
  \[
    \Delta C_{n,1},\Delta C_{n,2},\dots,\Delta C_{n,n}
  \]
where $\Delta C_{n,i}$ has length $\Delta s_{n,i}$, contains the position
vector $\vect{r}_{n,i}$, and
  \[
    \lim_{n\to\infty} \max(\Delta s_{n,i}) = 0
  \]
\end{definition}

\begin{theorem}
If $\vect{r}(t)$ is a parametrization of $C$ for $a \leq t \leq b$, then
  \[
    \int_C f(\vect{r})\dvar{s}
    =\int_{t=a}^{t=b} f(\vect{r}(t))\frac{ds}{dt}\dvar{t}
  \]
\end{theorem}

          \begin{problem}
            Evaluate $\int_C z + 2xy\dvar{s}$ where $C$ is the line segment
            from $(0,-1,3)$ to $(2,2,-3)$.
          \end{problem}

          \begin{solution}

          \end{solution}

          \begin{contributors}

          \end{contributors}

          \begin{problem}
            Prove that $\int_C xy\dvar{s}=\int_0^1 t^3\sqrt{1+2t}\dvar{t}$
            where $C$ is the parabolic arc
            on $y=x^2$ from $(0,0)$ to $(1,1)$.
          \end{problem}

          \begin{solution}

          \end{solution}

          \begin{contributors}

          \end{contributors}

\begin{definition}
The \textbf{line integral of a vector field} $\vect F$
over the curve $C$ is given by
  \[
    \int_C \vect F\cdot\dvar{\vect r} =
    \lim_{n\to\infty}\sum_{i=1}^n
    \vect F(\vect{r}_{n,i})\cdot\Delta\vect{C}_{n,i}
  \]
where for each positive integer $n$ we've defined a way to approximate $C$
with $n$ vectors
  \[
    \Delta \vect{C}_{n,1},\Delta \vect{C}_{n,2},\dots,\Delta \vect{C}_{n,n}
  \]
where $\vect{r}_{n,i}+\Delta \vect{C}_{n,i}=\vect{r}_{n,i+1}$
and
  \[
    \lim_{n\to\infty} \max(|\Delta \vect{C}_{n,i}|) = 0
  \]
\end{definition}

\begin{definition}
The line integral of a vector field $\vect F$ over the curve $C$
may be computed by
    \[
      \int_C \vect{F}\cdot\dvar{\vect r}
        =
      \int_C \vect{F}\cdot\vect{T}\dvar{s}
    \]
where $\vect T$ yields the unit tangent vectors to the curve $C$.
\end{definition}

\begin{definition}
If $\vect{r}(t)$ is a parametrization of $C$ for $a \leq t \leq b$, then
    \[
      \int_C \vect{F}\cdot\dvar{\vect r}
        =
      \int_{t=a}^{t=b} \vect{F}\cdot\frac{d\vect{r}}{dt}\dvar{t}
    \]
\end{definition}

          \begin{problem}
            Prove that
            $\int_C \<2x,y-x\>\cdot\dvar{\vect{r}}
              =
            \int_0^1 19t-5 \dvar{t}$
            where $C$ is the line segment given by the vector equation
            $\vect{r}(t)=\<1-2t,3t\>$ for $0\leq t\leq 1$.
          \end{problem}

          \begin{solution}

          \end{solution}

          \begin{contributors}

          \end{contributors}

\begin{remark}
  The work done by a force vector field $\vect{F}$ over the curve $C$
  is given by $\int_C\vect{F}\cdot\dvar{\vect{r}}$.
\end{remark}

          \begin{problem}
            Find the work done by the force vector field
            $\<-3y,3x\>$ moving a particle one rotation counter-clockwise
            around the unit circle $x^2+y^2=1$.
          \end{problem}

          \begin{solution}

          \end{solution}

          \begin{contributors}

          \end{contributors}

\begin{theorem}
  If $C$ may be split into two curves $C_1$ and $C_2$, then
  \[
    \int_C f\dvar{s}
      =
    \int_{C_1} f\dvar{s}
      +
    \int_{C_2} f\dvar{s}
  \]
  and
  \[
    \int_C \vect F\cdot\dvar{\vect r}
      =
    \int_{C_1} \vect F\cdot\dvar{\vect r}
      +
    \int_{C_2} \vect F\cdot\dvar{\vect r}
  \]
\end{theorem}

\begin{theorem}
  If $-C$ is the curve $C$ oriented in the opposite direction, then
  \[
    \int_C f\dvar{s}
      =
    \int_{-C} f\dvar{s}
  \]
  and
  \[
    \int_C \vect F\cdot\dvar{\vect r}
      =
    - \int_{-C} \vect F\cdot\dvar{\vect r}
  \]
\end{theorem}

          \begin{problem}
            Write a paragraph explaining why a negative appears in the
            previous theorem for the
            line integral of a vector field but not for an arclength
            line integral.
          \end{problem}

          \begin{solution}

          \end{solution}

          \begin{contributors}

          \end{contributors}


\section{The Fundamental Theorem for Line Integrals} %16.3

\begin{definition}
  If $\nabla f=\vect{F}$, then $f$ is a \textbf{potential function}
  for the \textbf{conservative field} $\vect{F}$.
\end{definition}

          \begin{problem} %x^2-3yz
            Prove that $\<2x,-3z,-3y\>$ is a conservative field by
            finding a potential function $f$ for it. Hint: such an $f$
            must satisfy that $f=x^2+\Phi_1(y,z)$, $f=-3yz+\Phi_2(x,z)$,
            and $f=-3yz+\Phi_3(y,z)$ for some functions $\Phi_i$. (Why?)
          \end{problem}

          \begin{solution}

          \end{solution}

          \begin{contributors}

          \end{contributors}

\begin{theorem}
  The Fundamental Theorem for Line Integrals:
  If $C$ is any smooth curve beginning at the point $A$ and ending at the
  point $B$, then
  \[
    \int_C \nabla f\cdot \dvar{\vect{r}} = \left[f\right]_A^B = f(B)-f(A)
  \]
\end{theorem}

          \begin{problem}
            Prove that if $C$ is any smooth \textbf{closed curve}
            (beginning and ending at the same point), then
            \[
              \int_C \nabla f\cdot \dvar{\vect{r}} = 0
            \]
          \end{problem}

          \begin{solution}

          \end{solution}

          \begin{contributors}

          \end{contributors}

          \begin{problem}
            Compute $\int_C\<4,z^2,2yz\>\cdot\dvar{\vect r}$ where
            $C$ is the curve given by
            $\vect{r}(t)=\<2^t,\sin (\pi t),4t^2\>$ for $0\leq t\leq 1$.
            Then compute $\int_{C'}\<4,z^2,2yz\>\cdot\dvar{\vect r}$ where
            $C'$ is the line segment starting at $(1,0,0)$ and ending
            at $(2,0,4)$.
          \end{problem}

          \begin{solution}

          \end{solution}

          \begin{contributors}

          \end{contributors}

          \begin{problem}
            Prove that if $f$ is a potential function for the vector field
            $\<P,Q,R\>$, then use the mixed derivative theorem to prove that
            $P_y=Q_x$, $P_z=R_x$, and $Q_z=R_y$.
          \end{problem}

          \begin{solution}

          \end{solution}

          \begin{contributors}

          \end{contributors}

\begin{theorem}
  $\vect{F}=\<P,Q,R\>$ is a conservative vector field if and only if
  $P_y=Q_x$, $P_z=R_x$, and $Q_z=R_y$.
\end{theorem}

          \begin{problem}
            Prove that
            $\int_C\<ye^{xy+z},xe^{xy+z},e^{xy+z}\>\cdot\dvar{\vect r}=0$
            where $C$ is the curve given by
            $\vect{r}(t)=\<\frac{1}{1+t^2},\cos t,e^{1-t^2}\>$
            for $-1 \leq t \leq 1$.
          \end{problem}

          \begin{solution}

          \end{solution}

          \begin{contributors}

          \end{contributors}


\section{Green's Theorem} %16.4

\begin{theorem}
  Let $C$ be the boundary of the region $R$ in the $xy$ plane oriented
  counter-clockwise, and let $\vect{F}$ be a two-dimensional vector field. Then
  \[
    \int_C \vect{F}\cdot\dvar{\vect{r}}
      =
    \iint_R \left(\frac{\p Q}{\p x}-\frac{\p P}{\p y}\right)\dvar{A}
  \]
\end{theorem}

          \begin{problem}
            Evaluate $\int_C\<x^2+y,x+y\>\cdot\dvar{\vect r}$ where
            $C$ is the boundary of the unit square oriented counter-clockwise.
          \end{problem}

          \begin{solution}

          \end{solution}

          \begin{contributors}

          \end{contributors}

          \begin{problem}
            Find the work done by a force vector field $\<y,2x\>$ moving an
            object around the
            boundary of the triangle with vertices $(1,2)$, $(-1,-2)$, and
            $(3,-2)$ oriented clockwise.
          \end{problem}

          \begin{solution}

          \end{solution}

          \begin{contributors}

          \end{contributors}


\end{document}