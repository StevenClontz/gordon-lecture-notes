\documentclass[letterpaper, twoside, 12pt]{book}
\usepackage{packet}


\begin{document}

\setcounter{chapter}{3}

\chapter{Packet 4.1: Sections 16.1-16.4}

\setcounter{chapter}{16}
\setcounter{section}{0}

\section{Vector Fields} %16.1

\begin{definition}
  A \textbf{vector field} assigns a vector to each point in 2D or 3D space.
    \[
      \vect{F}=
      \vect{F}(\vect{r})=
      \vect{F}(x,y)=
      \<P(x,y),Q(x,y)\>=
      \<P(\vect{r}),Q(\vect{r})\>=
      \<P,Q\>
    \]
    \[
      \vect{F}=
      \vect{F}(\vect{r})=
      \vect{F}(x,y,z)=
      \<P(x,y,z),Q(x,y,z),R(x,y,z)\>=
      \<P(\vect{r}),Q(\vect{r}),R(\vect{r})\>=
      \<P,Q,R\>
    \]
\end{definition}

          \begin{problem}
            Sketch the vector field $\vect{F}=\<x+y,2y\>$ for
            all $x\in\{0,1,2\}$ and $y\in\{0,1,2\}$.
          \end{problem}

          \begin{solution}

          \end{solution}

          \begin{contributors}

          \end{contributors}

\begin{remark}
  The gradient vector function
    \[
      \nabla f (x,y)
        =
      \<f_x(x,y),f_y(x,y)\>
    \]
    \[
      \nabla f (x,y,z)
        =
      \<f_x(x,y,z),f_y(x,y,z),f_z(x,y,z)\>
    \]
  is a vector field which yields normal vectors
  to the level surfaces of the function $f$.
\end{remark}

          \begin{problem}
            Compute $\nabla f$ for the function
            $f(x,y)=x^2-2xy+y$, and then
            sketch the vector field $\nabla f$
            all $x\in\{0,1,2\}$ and $y\in\{0,1,2\}$.
          \end{problem}

          \begin{solution}

          \end{solution}

          \begin{contributors}

          \end{contributors}


\section{Line Integrals} %16.2

\begin{theorem}
  Some vector functions which parameterize curves follow.
  \begin{itemize}
    \item
    A line segment beginning at $P_0$ and ending at $P_1$:
      \[
        \vect{r}(t) = \vect{P_0} + t\vect{P_0P_1}, 0\leq t\leq 1
      \]
    \item
    A circle centered at the origin with radius $a$:
      \[
        \vect{r}(t) = \<a\cos t,a\sin t\>, 0\leq t\leq 2\pi
        \text{ (full counter-clockwise rotation)}
      \]
      \[
        \vect{r}(t) = \<a\sin t,a\cos t\>, 0\leq t\leq 2\pi
        \text{ (full clockwise rotation)}
      \]
    \item
    A planar curve given by $y=f(x)$ from $(x_0,y_0)$ to $(x_1,y_1)$
      \[
        \vect{r}(t) = \<t,f(t)\>, x_0\leq t\leq x_1
        \text{ (left-to-right)}
      \]
      \[
        \vect{r}(t) = \<-t,f(-t)\>, -x_0\leq t\leq -x_1
        \text{ (right-to-left)}
      \]
    \end{itemize}
\end{theorem}

          \begin{problem}
            Give a vector function which parameterizes the line segment
            from the point $(0,3,-2)$ to the point $(4,-1,0)$.
          \end{problem}

          \begin{solution}

          \end{solution}

          \begin{contributors}

          \end{contributors}

          \begin{problem}
            Give a vector function which parameterizes the curve
            $y=x^3-2x$ from the point $(1,-1)$ to the point $(-1,1)$.
          \end{problem}

          \begin{solution}

          \end{solution}

          \begin{contributors}

          \end{contributors}

          \begin{problem}
            Give a vector function which parameterizes the curve
            $x^2+y^2=9$ from the point $(3,0)$ clockwise to the point $(0,-3)$.
          \end{problem}

          \begin{solution}

          \end{solution}

          \begin{contributors}

          \end{contributors}

\begin{definition}
The \textbf{line integral with respect to arclength} of a function of many
variables $f(\vect{r})$ along a curve $C$ is given by
  \[
    \int_C f(\vect{r})\dvar{s} =
    \lim_{n\to\infty}\sum_{i=1}^n f(\vect{r}_{n,i})\Delta s_{n,i}
  \]
where for each positive integer $n$ we've defined a way to partition $C$
into $n$ pieces
  \[
    \Delta C_{n,1},\Delta C_{n,2},\dots,\Delta C_{n,n}
  \]
where $\Delta C_{n,i}$ has length $\Delta s_{n,i}$, contains the position
vector $\vect{r}_{n,i}$, and
  \[
    \lim_{n\to\infty} \max(\Delta s_{n,i}) = 0
  \]
\end{definition}

\begin{theorem}
If $\vect{r}(t)$ is a parametrization of $C$ for $a \leq t \leq b$, then
  \[
    \int_C f(\vect{r})\dvar{s}
    =\int_{t=a}^{t=b} f(\vect{r}(t))\frac{ds}{dt}\dvar{t}
  \]
\end{theorem}

          \begin{problem}
            Evaluate $\int_C z + 2xy\dvar{s}$ where $C$ is the line segment
            from $(0,-1,3)$ to $(2,2,-3)$.
          \end{problem}

          \begin{solution}

          \end{solution}

          \begin{contributors}

          \end{contributors}

          \begin{problem}
            Prove that $\int_C xy\dvar{s}=\int_0^1 t^3\sqrt{1+2t}\dvar{t}$
            where $C$ is the parabolic arc
            on $y=x^2$ from $(0,0)$ to $(1,1)$.
          \end{problem}

          \begin{solution}

          \end{solution}

          \begin{contributors}

          \end{contributors}

\begin{definition}
The \textbf{line integral of a vector field} $\vect F$ over the curve $C$
is given by
    \[
      \int_C \vect{F}\cdot\dvar{\vect r}
        =
      \int_C \vect{F}\cdot\vect{T}\dvar{s}
    \]
where $\vect T$ is the unit tangent vector to the curve $C$ at the
position vector $\vect r_0$.
\end{definition}

\begin{definition}
If $\vect{r}(t)$ is a parametrization of $C$ for $a \leq t \leq b$, then
    \[
      \int_C \vect{F}\cdot\dvar{\vect r}
        =
      \int_{t=a}^{t=b} \vect{F}\cdot\frac{d\vect{r}}{dt}\dvar{t}
    \]
\end{definition}

          % \begin{problem}
          %   Prove that
          %   $\int_C \vect{F}\cdot\d{\vect{r}}
          %     =
          %   \int_0^1 (2t+1) $
          % \end{problem}

          % \begin{solution}

          % \end{solution}

          % \begin{contributors}

          % \end{contributors}

\end{document}