\documentclass[letterpaper, twoside, 12pt]{book}
\usepackage{packet}

\parskip=1em


\begin{document}

{\Large Notes on MATH 2630 Final Exam (Spring 2015)}

The final exam will take place during one of the following periods:

\begin{itemize}
\item For MWF 10am class: Wed 2015-05-06 0800-1030
\item For MWF 12noon class: Tues 2015-05-05 1200-1430
\end{itemize}

If you prefer, you may take the final exam during the other section's
timeslot. Unless many students in one section take the exam during the
other section, these exams will be held in the usual classroom.
You must take your exam during one of these times.

The structure of the exam follows:

\begin{itemize}
  \item 6 questions based on Tests 1-4
  \item 3 questions based on Problems in Packets 1-4
  \item choose 1 of 2 new types of questions based on Packets 1-4
\end{itemize}

Each question is worth 10 points, for a total of 100 points. This total
will be scaled to match the weight of your final exam, with rounding up to
the next integer (see syllabus for details on final exam weighting).

The final exam is run exactly like the Individual Tests, including the open
notes/computers policy, and with no collaboration allowed between classmates
and no communication allowed with others outside the classroom. You will
of course have much more time to complete this exam (150 minutes for only twice
the questions asked on a 40 minute test).

\textbf{Suggestions for studying:} Focus first on the questions asked
on Tests. Then go over as many Problems from the Packets as possible.
It's probably not useful to attempt to cram for the new types of questions,
but you will choose which of the two questions you want to respond
to.

\end{document}