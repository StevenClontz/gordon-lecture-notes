\documentclass[12pt]{exam}

\newcommand{\ds}{\ensuremath{\displaystyle}}

\usepackage{amsmath,amsfonts, amsthm}
\usepackage{multicol}
\usepackage{multirow}
\usepackage{harpoon}
\renewcommand{\arraystretch}{1.5}

\newcommand{\harpvec}[1]{\overrightharp{\ensuremath{\mathbf{#1}}}}
\newcommand{\vect}[1]{\harpvec{#1}}
\newcommand{\<}{\langle}
\renewcommand{\>}{\rangle}

\printanswers

\begin{document}

\begin{center}
\fbox{\fbox{\parbox{5.5in}{\centering
Calculus III - Spring 2015 - Mr. Clontz - Test 4
}}}
\end{center}
\vspace{0.1in}
\makebox[\textwidth]{
  Name:\enspace\hrulefill\hrulefill\hrulefill\space
  Class:\enspace\hrulefill\space
  Type:\enspace\hrulefill
}

\vspace{12pt}

\begin{itemize}
  \item This exam is open-``everything'', provided you do not plagiarize.
        \textit{Do not leave any unsupported answers!}
  \item Write your solutions so that a fellow student who has a perfect
        understanding of previous math classes and packets but
        has never seen that type of problem before could follow your work.
  \item Individual Test: You will have 40 minutes to complete this test
        on your own. You may not communicate with anyone during this period.
  \item Group Test: You will have 40 minutes to complete an identical test.
        You may collaborate with your group members during this time. All
        solutions must still be written by yourself, and may not be directly
        copied from another student.
\end{itemize}

\newpage

\begin{questions}

\setcounter{question}{0}

\question[6]
  Prove that \[\int_C x+z\,d{s}=\int_0^1 6-3t\,d{t}\]
  where \(C\) is the line segment from \((1,-2,1)\) to \((2,0,-1)\).

\newpage

\question[6]
  Prove that
    \[
      \int_C \<-y^2,x\>\cdot\,d\vect{r}
        =
      \int_0^{2\pi} \sin^3 t + \cos^2 t\,dt
    \]
  where \(C\) is the full counter-clockwise rotation of the circle
  \(x^2+y^2=4\). (Hint: this is \textit{not} a conservative field.)

\newpage

\question[6] % f=x^2y-3y^2z
  Prove that
    \[
      \int_C \<2xy,x^2-6yz,-3y^2\>\cdot\,d\vect{r}
        =
      -34
    \]
  where \(C\) is a curve beginning at \((1,2,3)\) and ending at
  \((4,0,1)\). (Hint: this \textit{is} a conservative field.)

\newpage

\question[6]
  Let \(C\) be the boundary of a surface \(S\).
  Use Stokes' Theorem (section 16.8) to prove that
    \[
      \iint_S \<2z,2x,2y\>\cdot\,d\vect{\sigma}
        =
      \int_C \<x-y^2,y-z^2,z-x^2\>\cdot\,d\vect{r}
    \]

\newpage

\question[6]
  Let \(\vect{E}=\<E_1,E_2,E_3\>\) be an electric vector field defined for
  each point in the interior \(D\) of a closed surface \(S\).
  By Gauss's Law, if there is a total charge of \(Q\) contained
  by \(S\), then the total
  electric flux across the surface \(S\) is given by
    \[
      \iint_S \vect{E}\cdot\,d{\sigma} = \frac{Q}{\epsilon_0}
    \]
  where \(\epsilon_0\) is known as the electric constant (roughly
  \(8.85\times 10^{-12}\), not that it matters).

  Prove that the total charge \(Q\) may be computed as
    \[
      \epsilon_0\iiint_D
        \frac{\partial E_1}{\partial x} +
        \frac{\partial E_2}{\partial y} +
        \frac{\partial E_3}{\partial z}
      \,d{V}
    \]
  (Of course, no knowledge of physics is required to answer this question.)



\end{questions}

\end{document}