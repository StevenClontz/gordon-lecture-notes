\documentclass[letterpaper, twoside, 12pt]{book}
\usepackage{packet}


\begin{document}

\setcounter{chapter}{1}

\chapter{Part 2.4: Sections 14.7-14.8}

\setcounter{chapter}{14}
\setcounter{section}{6}

\section{Maximum and Minimum Values} %14.7

\begin{definition}
  Let $f$ be a function of many variables defined near the point $P_0$.
  Then $f$ has a \textbf{local maximum} $f(P_0)$ at $P_0$ if $f(P_0)$ is the
  largest value of $f$ near $P_0$, and
  $f$ has a \textbf{local minimum} $f(P_0)$ at $P_0$ if $f(P_0)$ is the smallest
  value of $f$ near $P_0$.
  (Local maxima and minimal are also known as local extreme values or
  local extrema.)
\end{definition}

\begin{definition}
  If $P_0$ is a point in the domain of $f$ and
    \[
      \nabla f(P_0) = 0 \text{ or } \nabla f(P_0) \text{ DNE}
    \]
  then $P_0$ is called a \textbf{critical point}.
\end{definition}

\begin{theorem}
  Critical points of a two-variable function occur when the tangent plane
  is horizontal (because $\<0,0,-1\>$ is a normal vector)
  or the tangent plane does not exist.
\end{theorem}

\begin{theorem}
  The local maximum and minimum values of a function always
  occur at critical points.
\end{theorem}

          \begin{problem}
            Prove that $f(x,y)=x^2+16y^2$ has exactly one local extreme value
            value by showing that $(0,0)$ is the only critical point
            for $f$, and then showing that $f(0,0)$ is the minimum value
            of the function.
          \end{problem}

          \begin{solution}

          \end{solution}

\begin{remark}
  By plotting the graph of $f$ in the previous problem, you can see
  that $(0,0)$ yields the lowest point on the surface.
\end{remark}

\begin{definition}
  The \textbf{saddle points} of $f$ are the critical points which don't yield local extreme values.
\end{definition}

          \begin{problem}
            Prove that $(0,0)$ is a saddle point of the function
            $f(x,y)=4x^2-9y^2$ by first showing that it is a critical point,
            and then considering the function $f$ restricted to the curves $y=0$
            and $x=0$ in the $xy$ plane.
          \end{problem}

          \begin{solution}

          \end{solution}

\begin{remark}
  The term ``saddle point'' comes from the fact that the graph near a
  saddle point often looks like a saddle (such as in the previous problem).
\end{remark}

\begin{definition}
  The \textbf{discriminant} of a differentiable two variable function
  $f$ is the function
  \[
    f_D
      =
    \begin{array}{|cc|}f_{xx}&f_{xy}\\f_{yx}&f_{yy}\end{array}
      =
    f_{xx}f_{yy} - f_{xy}^2
  \]
\end{definition}

          \begin{problem}
            Compute the discriminant of the function
            $f(x,y)=3x^2y-2y^3+4x$.
          \end{problem}

          \begin{solution}

          \end{solution}

\begin{theorem}
  The \textbf{Second Derivative Test} for two-variable functions gives
  a way to (sometimes) determine if a critical point either yields a
  local maximum, a local minimum, or a saddle point.
  Let $(a,b)$ be a critical point of of $f$ where $\nabla f$ is defined.
    \begin{itemize}
      \item If $f_D(a,b)>0$ and $f_{xx}(a,b)<0$,
            then $f(a,b)$ is a local maximum.
      \item If $f_D(a,b)>0$ and $f_{xx}(a,b)>0$,
            then $f(a,b)$ is a local minimum.
      \item If $f_D(a,b)<0$,
            then $f$ has a saddle point at $(a,b)$.
      \item If $f_D(a,b)=0$,
            then the test is inconclusive.
    \end{itemize}
\end{theorem}

          \begin{problem}
            Prove that $f_{xx}$ could be replaced with $f_{yy}$ in the
            Second Derivative Test by showing that if $f_D(a,b)>0$,
            then $f_{xx}(a,b)$ and $f_{yy}(a,b)$ are either both positive
            or both negative.
          \end{problem}

          \begin{solution}

          \end{solution}

          \begin{problem}
            In an earlier problem we found that $f(x,y)=x^2+16y^2$ has exactly
            one critical point $(0,0)$. Use the Second Derivative Test
            to show that $f(0,0)$ is the minimum value of the function.
          \end{problem}

          \begin{solution}

          \end{solution}

          \begin{problem}
            Identify all the critical points for
            $f(x,y)=x^3-6xy+\frac{3}{2}y^2-1$, then use the Second Derivative
            Test to label each critical point as yielding a local minimum,
            a local maximum, or a saddle point.
          \end{problem}

          \begin{solution}

          \end{solution}

\begin{definition}
  Let $f$ be a function of many variables.
  Then $f$ has an \textbf{absolute maximum} $f(P_0)$ at $P_0$ if $f(P_0)$ is
  the largest value in the range of $f$, and
  $f$ has an \textbf{absolute minimum} $f(P_0)$ at $P_0$ if $f(P_0)$ is
  the smallest value in the range of $f$.
  (Absolute maxima and minimal are also known as the absolute extreme values or
  absolute extrema.)
\end{definition}

\begin{theorem}
  A continuous function $f$ restricted to a closed and bounded domain $D$
  always has an absolute minimum and absolute maximum value.
\end{theorem}

\begin{theorem}
  The only possible points which can yield the absolute value of a function
  $f$ of two variables $x,y$ on a restricted domain $D\subseteq \mathbb R^2$
  are:
  \begin{itemize}
    \item Critical points of $f$ inside $D$
    \item Critical points of $f$ restricted to the boundary of $D$
    \item Corners on the boundary of $D$
  \end{itemize}
  The absolute maximum and absolute minimum values may be computed by
  plugging in all of these candidates into $f$.
\end{theorem}

\begin{remark}
  The previous theorem works because it checks all the local extreme values
  against each other to find the absolute largest and smallest value.
\end{remark}

          \begin{problem}
            Find the absolute maximum and minimum value of
            $f(x,y)=x^2+y^2-2x-2y$ restricted to the region bounded by
            the triangle with vertices
            $(0,0)$, $(2,4)$, and $(2,0)$. (Hint: this
            triangle is given by the curves $y=0$, $y=2x$, and
            $x=2$.)
          \end{problem}

          \begin{solution}

          \end{solution}

          \begin{problem}
            Find the absolute maximum and minimum value of
            $f(x,y)=2xy$ restricted to the region bounded by
            the circle $x^2+y^2=4$. (Hint: find a vector equation $\vect{r}(t)$
            for this circle and find the critical points of $f(\vect{r}(t))$.)
          \end{problem}

          \begin{solution}

          \end{solution}


\section{Lagrange Multipliers} %14.8

          \begin{problem}
            A rancher wants to enclose a rectangular area using the
            straight edge of a cliff on one side, and barbed wire on
            the other three sides. If the rancher wants to maximize
            the area of this rectangle, what are the dimensions of the
            fence and the maximized area? In other words, find $x,y$ which
            maximize $A(x,y)=xy$ given the constraint $2x+y=100$, and
            the value of $A$ for those values.
          \end{problem}

          \begin{solution}

          \end{solution}

          \begin{problem}
            Plot the constraint $2x+y=100$ from the previous problem in
            the $xy$-plane, along with the level curves of $A(x,y)=xy$
            for $k=750,1000,1250,1500$. (Make sure the image includes
            $x\in[-100,100]$ and $y\in[-100,100]$.)
          \end{problem}

          \begin{solution}

          \end{solution}

\begin{remark}
  Notice that the point $(x,y)$ which maximizes the value of $A(x,y)$
  is where the constraint $2x+y=100$ and the level curve $xy=1250$
  share the same tangent line (and therefore normal vectors).
\end{remark}

\begin{theorem}
  A function
  $f(P)$ of many variables, constrained by the requirement
  $g(P)=k$ for some function $g$ and constant $k$, is maximized or
  minimized at a point $P_0$ where the normal vector to the
  level curve/surface $f(P)=f(P_0)$ is parallel to the normal vector
  to the curve/surface $g(P)=k$. Put another way:
  \[
    \nabla f = \lambda(\nabla g)
  \]
\end{theorem}

\begin{theorem}
  The \textbf{Method of Lagrange Multiplers} for two-variable functions
  states that to maximize/minimize $f(x,y)$ on the constraint $g(x,y)=k$,
  you should solve the system of equations
    \[
      f_x(x,y)=\lambda g_x(x,y)
    \]
    \[
      f_y(x,y)=\lambda g_y(x,y)
    \]
    \[
      g(x,y)=k
    \]
  where $\lambda$ is an unknown real number,
  and testing all solutions of $x,y$ to find the maximum
  and minimum values of $f$.
\end{theorem}

          \begin{problem}
            Use the Method of Lagrange Multipliers to solve the first
            problem of this section. (Tip: to start, use the first
            two equations and eliminate the variable $\lambda$ since
            it's not needed for the solution.)
          \end{problem}

          \begin{solution}

          \end{solution}

          \begin{problem}
            Find the minimum surface area of a right circular cylinder
            with volume equal to $432\pi$ cubic units.
            (Hint: $V=\pi r^2h$ and $SA=2\pi r(r+h)$.)
          \end{problem}

          \begin{solution}

          \end{solution}

          \begin{problem}
            OPTIONAL.
            A river of constant width $10\sqrt{3}\approx 17.32$ meters
            flows $40$ meters per second from north to south. A swimmer
            on the west side of the river can swim at a constant
            $20$ meters per second through still waters, but since the flow
            of the river is faster than her top speed, this swimmer will
            unavoidably be pushed downstream if she tries to swim across.

            Use the Method of Lagrange Mutlipliers to prove that if
            the swimmer sets an angle of $\frac{\pi}{6}=30^\circ$ north of
            east, then she will minimize the distance she is pushed downstream
            as she swims from the west riverbank to the east riverbank.
            (Hint: define $x(\theta,t)$ to be the distance she travels east
            after $t$ seconds if she sets the angle $\theta$, and define
            $y(\theta,t)$ to be the distance she travels south
            after $t$ seconds if she sets the angle $\theta$. Then your
            constrant is that $x(\theta,t)$ should be the width of the river,
            although that's actually not needed to solve the puzzle...)
          \end{problem}

          \begin{solution}

          \end{solution}

\begin{theorem}
  The \textbf{Method of Lagrange Multiplers} for three-variable functions
  states that to maximize/minimize $f(x,y,z)$ on the constraint $g(x,y,z)=k$,
  you should solve the system of equations
    \[
      f_x(x,y,z)=\lambda g_x(x,y,z)
    \]
    \[
      f_y(x,y,z)=\lambda g_y(x,y,z)
    \]
    \[
      f_z(x,y,z)=\lambda g_z(x,y,z)
    \]
    \[
      g(x,y,z)=k
    \]
  where $\lambda$ is an unknown real number,
  and testing all solutions of $x,y,z$ to find the maximum
  and minimum values of $f$.
\end{theorem}

          \begin{problem}
            Find the maximum volume of a rectangular box without a lid
            which uses $48$ square units of material.
          \end{problem}

          \begin{solution}

          \end{solution}

          \begin{problem}
            Find the highest and lowest points which lay on the curve of intersection for the cylinder $x^2+y^2=8$ and the plane
            $2x+2y+z=16$.
          \end{problem}

          \begin{solution}

          \end{solution}

\end{document}